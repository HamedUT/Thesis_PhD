\section{Conclusion}
\label{sec:conclusion}

Three significant advancements were realized for the first time in the scope of this chapter. Firstly, the estimation of \gls{lac} and fatigue was successfully executed using both deep learning and machine learning models. Secondly, the creation of a participant-independent \gls{lac} estimation model marked a noteworthy achievement. Lastly, an analysis of the utilization of kinematic parameters in the estimation of a physiological parameter, \gls{lac}, was conducted.

This chapter highlights the potential of utilizing data from body-mounted \gls{imu}s on sport horses instead of invasive to measure \gls{lac} values to evaluate fatigue levels during exercise. In addition, the incorporation of a \gls{lac} monitor and using its output as an input for our models enhances the performance of the models. Besides the measurement of physiological parameters, the application of \gls{imu}s for sport horses can supply more objective tools for detecting fatigue and benefiting riders and trainers. This can mitigate excessive fatigue, thereby reducing injury rates and improving the well-being of horses. In addition, the chapter outcomes could be used to monitor the fatigue levels of horses during exercise, to ensure that horses are not pushed beyond their capacities when they are too fatigued, and to reduce the risk of injuries. For broader applicability, future studies will explore exercise data across various disciplines and additional body locations for \gls{imu} placement, aiming to develop a more comprehensive and versatile model.
