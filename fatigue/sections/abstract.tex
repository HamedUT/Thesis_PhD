\begin{quote}

Exercise-induced fatigue is a complex phenomenon that can significantly impact the performance, health, and welfare of sport horses. Traditional methods for assessing fatigue in sport horses, such as plasma \gls{lac} measurement, can be invasive and cumbersome. In this chapter, we propose the use of body-mounted \gls{imu} and a \gls{hr} monitor as a non-invasive approach for fatigue assessment by estimating \gls{lac} of sport horses throughout the exercise. \gls{lac} estimation models were trained using signal-based features and kinematic parameters extracted from \gls{imu}s. As an outcome, the accuracy of the best-performing model based on two \gls{imu}s and \gls{hr} was 0.11 \gls{mmol/L} for \gls{rmse} and 4.89\% for \gls{mape}. This approach holds the capability to accurately monitor fatigue in horses during training, optimize exercise intensity, and prevent injuries.  The application of \gls{imu} for fatigue assessment in sport horses is a promising new approach that has the potential to enhance their health and welfare.

\end{quote}

\blfootnote{This chapter is a revised and extended version of the work originally published as \textit{'A Non-Invasive Lactate Estimation Using Wearable Sensors for Remote Fatigue Assessment in Horses'} in the Proceedings of The 22nd International Conference
on Pervasive Computing and Communications (PerCom 2024).}