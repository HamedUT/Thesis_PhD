\section{Horses}
\label{sec:Horses}

Table \ref{tab:IMU info on horses} in Appendix \ref{appendix:appendix_placeholder} indicates the information of the horses with an assigned code, height, weight, age, sex, breed, competing level, and the \gls{imu}s mounted on the body during \gls{set} measurements. To ensure the preservation of competitive interests, the horses involved in this study were assigned unique identifiers instead of using their actual identities. This approach was implemented to maintain the anonymity of the horses and to respect the concerns of their owners and caretakers. All owners of the participating horses granted written consent for the utilization of their horse data for research purposes. The Animal Ethics Committee of Utrecht University granted ethical permissions specifically for the measurement of Friesian horses. However, it was determined by the Committee that ethical approval was not necessary for measuring the remaining horses as it did not meet the criteria for an animal experiment according to Dutch law.

Data were collected from eighty-eight ridden sport horses. The sport horses trained and competed in different disciplines, forty-four in eventing, twenty-eight in endurance racing, ten in showjumping, and six in dressage. Among the eventing horses, sixteen were of the Friesian breed, while others were of the Warmblood breed. All the horses performed the designated \gls{set} once, except three eventing horses, namely E11, E15, and E17, which performed the \gls{set} three times on separate days for purposes unrelated to this thesis (heat acclimatization purposes). Further information regarding the designated \gls{set}s can be found in Section \ref{sec:trainingprotocol}. The average age of all the horses was 11.9 $\pm$ 2.9 years, while for eventing, dressage, and endurance horses were 11.1 $\pm$ 3.0, 12.7 $\pm$ 1.6, and 13.0 $\pm$ 2.1 years, respectively. All the horses were examined for lameness pre- and post-SET by a veterinarian. The ones that presented lameness during the examinations were excluded from this study.


The inclusion criteria were horses that either performed on an international competition level or were selected for the final studbook approval test. All the horses were competing under \gls{fei} rules. Six eventing horses were qualified for the summer Olympic games, while other eventing horses were competing at the international level from Concours Complet International 2-star (CCI2*) to 5-star (CCI5*). The dressage horses were competing at the Grand Prix level and were all qualified for the summer Olympic games in Tokyo 2020. The endurance horses were also competing at the international level, Concours de Raid D'Endurance International 2-star (CEI2*) and 5-star (CEI5*). Moreover, all the horses were ridden by their specific owner or rider for \gls{set}s.