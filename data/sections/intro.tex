\section{Introduction}
\label{sec:intro_introduction}

Longitudinal monitoring and data collection from horses offer a plethora of benefits across various domains. In equine health, the continuous tracking of vital signs, activity levels, and behavioral patterns enables early detection of anomalies, leading to prompt intervention and improved overall well-being. This approach aids veterinarians in making informed decisions and designing personalized treatment plans. Moreover, in the realm of performance and training, longitudinal data helps trainers and riders understand a horse's progress, identify training strategies that yield optimal results, and prevent overexertion or injury.

From a research perspective, longitudinal data provides insights into the long-term effects of different management practices, nutrition, and environmental factors on equine health and behavior. Harnessing this data can contribute to advancements in veterinary science and the betterment of equine practices. Furthermore, in industries like competitive sports and breeding, longitudinal monitoring offers a comprehensive understanding of genetic traits, reproductive cycles, and performance potentials, aiding in making informed breeding and training decisions. Overall, the benefits of longitudinal monitoring and data collection from horses extend to their health, performance, research, and industry practices, creating a holistic approach to equine well-being and advancement.

