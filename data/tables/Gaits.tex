\begin{table}[!htb] 
\centering
\caption{Description of gaits used in this book. The type of gait indicates if the horse can perform the gait without (natural) or with (artificial) learning. The footfall patterns are illustrated in Figure \ref{fig:horseshoe}}% Add 'table' caption
\resizebox{\linewidth}{!}{%
\begin{tabular}{p{1cm}p{1.5cm}p{10cm}}
\toprule
\textbf{Gait} & \textbf{Type} & \textbf{Definition} \\
\midrule 

Walk  & Natural  & A lateral symmetrical gait consists of four-beat rhythm with the hoof land and take off in the following sequence: left hind hoof, left front hoof, right hind hoof, right front hoof.\\

Trot  & Natural  & A diagonal symmetrical gait consists of two-beat diagonal rhythm. The right front hoof and left hind hoof rise and fall together alternately with the diagonal pair left front hoof and right hind hoof. \\

Pace  & Natural  & A lateral gait characterized by a two-beat lateral pattern. In a pace gait, the horse moves its hooves on one side of its body together, followed by the hooves on the opposite side. It is a gait commonly associated with certain horse breeds, such as Standardbred trotters and Icelandic horses, and is known for its smoothness and speed. \\

Canter  & Natural  & An asymmetric gait, consists of three-beat rhythm with the following sequence: left hind hoof, right hind hoof and left front hoof simultaneously, right front hoof for right lead canter. For left lead canter,  right hind hoof, right front hoof and left hind hoof simultaneously, left front hoof. \\

Tölt  & Natural  & A four-beat gait, hooves strike the ground in a specific sequence: right hind, right front, left hind, left front. It is a distinctive gait predominantly associated with Icelandic horses, known for its smooth and comfortable motion. It lacks a suspension phase, distinguishing it from the trot, and can be performed at various speeds. \\

Passage  & Artificial  & A measured, very collected, elevated and cadenced trot \cite{clayton_2019_a}. One of the important gait types that can be learned by dressage horses. \\

Piaffe  & Artificial  & A highly collected, cadenced, elevated diagonal movement in place or nearly in place \cite{clayton_2019_a}, specifically for dressage horses.\\
\bottomrule 
\label{tab:gait}
\end{tabular}}
\end{table}