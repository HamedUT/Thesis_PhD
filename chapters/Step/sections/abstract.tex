\begin{quote}
Accurate calculation of temporal stride parameters is essential in horse gait analysis. A prerequisite for calculating these parameters is identifying the exact timings of gait events, namely, hoof-on and hoof-off moments. A hoof-mounted \gls{imu} can be used to identify these moments accurately, yet this approach is often impractical due to the vulnerability of IMU to the impacts during locomotion. In this study, we investigated the possibility of accurately estimating the gait events using the signals of an \gls{imu} mounted on a less vulnerable location, such as a limb or upper body. To achieve the goal, we equipped IMUs on limbs, withers, and the sacrum of the horses and measured them during different gaits. Then, we estimated the gait events timings by training recurrent neural network models on the output signals of each \gls{imu}. Finally, we evaluated the models by comparing their results to the gait events timings labeled from hoof-mounted \gls{imu}s. The best-performing model represented the best location (between the limbs, withers, and the sacrum) for gait event estimation. Compared to the previous studies, our models yielded higher accuracy and were more generic by supporting more gait types. In conclusion, accurate calculation of temporal stride parameters is feasible by estimating gait event timings using an \gls{imu} mounted on less vulnerable body locations.
\end{quote}

\blfootnote{This chapter is a revised and extended version of the work originally published as \textit{'Accurate horse gait event estimation using an inertial sensor mounted on different body locations'} in the Proceedings of 2022 IEEE International Conference on Smart Computing (SMARTCOMP).}