\section{Discussion}
\label{sec:discussion}

In this chapter, four outcomes have been achieved for the first time in equine literature. First, the models outperformed the previous studies in terms of estimation accuracy. Second, using the signals from upper body \gls{imu}s, we accurately estimated hoof-on and hoof-off moments. Third, by considering the different patterns between gaits, hoof-on and hoof-off moments during the five gaits were accurately estimated by developing single models that support all gaits. Lastly, a deep learning approach was implemented for hoof moment estimation to ease the way for developing automatic and real-time applications.

\subsection{Analysis of the models and the results}

%Adding the Euclidean norm of acceleration and angular velocity signals to the model input signals helps to eliminate the disorientation placement of IMUs. In addition, it 

Since the models were fed with 200 Hz data, the output was also 200 Hz. Therefore, the lowest error value is one timestep, which equals five milliseconds. For models based on front limb and withers \gls{imu}s, we achieved mean accuracies lower than a full timestep (< 5 ms), indicating the model's capability to estimate many testing samples without error.

%\subsection{Comparison between the performances of  front and hind hoof on/off models}

According to Figure \ref{models_result_fig} and Table \ref{results_of_models}, the model based on limbs \gls{imu}s data yielded the highest precision (i.e., lowest standard deviation of errors), while the models based on the right front limb (0 ms) and withers \gls{imu}s (-0.2 ms) were the most accurate (i.e., lowest mean of errors) for hoof-off and hoof-on. This result indicates that calculating the temporal stride parameters using the front hoof-on and hoof-off estimated moments is more accurate than the hind hoof-on and hoof-off (6.5 and 5.6 ms for the model based on hind limb \gls{imu}), independent of \gls{imu} placement (on a limb or upper body). Even with lower accuracy of the hind hoof-on and hoof-off models, their errors for calculating the stride duration meet the criteria for distinguishing between different degrees of hind limb lameness \cite{weishaupt_2010_compensatory} as well as for front limb lameness \cite{weishaupt_2006_compensatory}. In addition, calculating stride duration from front hoof-on and hoof-off models is sufficient for detecting training impacts on racehorses (< 0.6 ms) \cite{parkes_2019_the}.

%\subsection{Comparison of gaits and IMU placements}

As shown in the four bottom plots of Figure \ref{models_result_fig}, the accuracy and precision of the models are correlated inversely with speed and are lower in artificial gaits compared to natural gaits. The reason might be due to more sudden lower limb movements (creating a more complex pattern) during gaits with higher speed values. Specifically, the complex patterns are more present during piaffe and passage, where the dressage horses perform special movements focusing on the lower limbs.

As displayed in the top plot of Figure \ref{models_result_fig}, the models based on the right front and the hind limb \gls{imu}s performed better than the models based on the withers and the sacrum \gls{imu}s, respectively. From the vulnerability aspect, equipping a limb, withers, or sacrum for the measurement is more secure than attaching an \gls{imu} to the hoof. Moreover, from a practical standpoint, equipping a limb with \gls{imu} is more straightforward than mounting an \gls{imu} on top of the withers or sacrum. Similar to the data collection section and Figure \ref{horse}, the limb \gls{imu} can be fitted in a pocket, where the pocket is wrapped around a limb or a tendon boot (horse boot) using hook-and-loop straps. Fixing the pocket on the limb and placing the \gls{imu} inside the pocket is feasible with simple guidance for an average user. On the other hand, mounting an \gls{imu} on withers or sacrum is usually done by sticking a double adhesive tape; However, if the measurement duration gets prolonged, the adhesive tape becomes loose due to the horse's body sweat, and the sensor falls off. Moreover, finding the anatomical location of the sacrum requires equine anatomical knowledge. Therefore, the practical advantage of using limb \gls{imu} for estimation adds up to its better performance result compared to the upper body \gls{imu}. %The advantage of placing the \gls{imu} on a limb than on the sacrum and withers was also discussed in the machine learning studies within equine gait literature \cite{darbandi_2021_using,darbandi_2022_a}.

%\subsection{Comparison with state-of-the-art}

The models' performance results were validated by applying two state-of-the-art methods Bragan\c{c}a et al. \cite{adsd1} and Sapone et al. \cite{sapone_2020_comparison}. In the first study, accelerometer and gyroscope signals were processed during walk and trot (surface type was not specified) \cite{adsd1}. In the latter, an estimation model was developed by implementing a discrete wavelet analysis on the longitudinal axis of the gyroscope mounted on the right front distal limb while the horses were trotting on a treadmill, which achieved an accuracy and precision of 3.7 ± 17.0 ms for hoof-on and 0.3 ± 15.4 ms for hoof-off (by considering the mean trotting stride duration was 680 ms in the study) \cite{sapone_2020_comparison}. The accuracy results from the studies are presented in Figure \ref{literature_compare_fig} and Table \ref{literature_compare_}. Except for hind hoof-on during walk and trot in \cite{adsd1}, our models outperformed both studies. In addition, our model based on withers \gls{imu} presented better accuracy and precision in both right front hoof-on and -off during walk and trot, compared to the results from state-of-the-art right front limb-based methods. 

The hoof estimation studies that used \gls{imu} have been focused on walk or trot \cite{starke_2012_accuracy,adsd1,olsen_2012_accuracy,sapone_2020_comparison}. In comparison, our models included more gaits. In addition, a large number of strides were used for model development compared to the small number of samples used in previous studies. Furthermore, the data in this study was measured from a training field, while the mentioned studies collected data in laboratory settings or treadmills.

\subsection{Considerations and limitations}

We did not consider the data from the left limbs since all quadrupedal vertebrates perform bilateral movement symmetry between front limbs and hind limbs \cite{abourachid_2003_a}. Therefore, the \gls{imu}s were placed only on the right limbs and hooves for this study.

It is worth mentioning the disadvantage of the models; the input window to the models must contain a hoof-on or a hoof-off. The models will select hoof-on and hoof-off moments regardless of the availability of hoof-on and hoof-off moments in the window, and in the case of no moment in the window, the model output would be counted as a false positive. A possible solution can be detecting the window containing the hoof-on or hoof-off moment using the Euclidean signal peaks \cite{tijssen_2020_automatic}, and then selecting the detected window for the current moment estimation model.

Depending on the surface type, the signal pattern of hoof \gls{imu} might be different. As a limitation of the current study, the data was collected from horses performing gait on a soft surface (sand mixed with fiber) only and the effects of hard surfaces were not analyzed on the models.


