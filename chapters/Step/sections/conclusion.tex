\section{Conclusion}
\label{sec:conclusion}

This chapter demonstrates that by mounting a single \gls{imu} on less vulnerable body locations and utilizing its generated signals across five different gaits, it is feasible to calculate temporal stride parameters with a level of accuracy exceeding the requirements for detecting variations in lameness degrees and assessing training impacts on performance. The estimation models established in this study can be integrated into the \gls{imu} for real-time, automated calculation of temporal stride parameters. The practical implications of this research can assist both researchers and equestrians in measuring gait parameters by securely affixing an \gls{imu} to a horse's body, preferably on a front limb for enhanced accuracy. The outcome models have the potential for further development through the inclusion of data from diverse horse breeds and various surface types.