\section{Lessons Learned}
\label{sec:lessons_learned}

During the course of this research, we have gathered insights and lessons learned, which we summarize below:

\begin{itemize}[label={$\bullet$}, leftmargin=*]

\item{\textit{Low-dimensional, perceivable, and tangible parameters may not always hold significance for machine learning algorithms}}

Humans can readily comprehend low-dimensional features and parameters, such as stride duration or horse speed, which are pertinent to fitness assessment. However, there may exist more critical parameters that offer a more accurate reflection of fitness. Given the limitations of human perception concerning high-dimensional parameters, there is a common reliance on comprehensible and tangible parameters, particularly in traditional studies where machine learning algorithms are not employed. 

Take, for example, the magnitude of the first coefficient of the x-axis in the withers acceleration signal as a vital parameter for speed estimation (see Table \ref{tablefeature}). This parameter, though crucial, eludes human researchers during conventional studies that do not incorporate pattern recognition and machine learning algorithms. In conclusion, sole reliance on existing literature is insufficient for a comprehensive study. To achieve a more refined understanding, complex and state-of-the-art algorithms are imperative.

\item{\textit{Simple algorithms can outperform complex ones, as complexity doesn't guarantee better performance}}

In the domain of machine learning, a compelling principle emerges: the potency of simplicity. It is a phenomenon well-observed that straightforward and uncomplicated machine learning algorithms can, at times, outshine their complex counterparts. While complexity may imply sophistication, it does not inherently translate to improved performance. In fact, the intricacies of complex algorithms can often lead to overfitting, causing models to perform admirably on training data but falter when confronted with new, unseen data. 

Simpler algorithms, on the other hand, are endowed with the virtue of transparency and ease of interpretation. Their ability to generalize patterns from data without being bogged down by excessive complexity makes them a valuable asset in various machine learning endeavors. In essence, the mantra remains clear: simplicity can yield remarkable results, and the quest for more complex solutions is not always the path to superior performance.

\item{\textit{Data labeling sometimes needs a human touch}} 

Automated data labeling is feasible for many tasks, especially when dealing with large datasets, but its effectiveness depends on the nature and the quality of the data. Additionally, the specific domain and application will influence the complexity of the automated labeling process. In the case of labeling hoof \gls{imu} signals to detect hoof-on and hoof-off, there were numerous instances where human decision was required (\ref{chapter:Step}).

\item{\textit{Exploring new frontiers often means dealing with tough challenges}} 

Numerous challenges exist in a field where artificial intelligence has not fully developed yet. It is apparent that artificial intelligence has made great strides in human sports and fitness studies, however, it has rarely been used in equine research. This posed a unique problem as it made it hard to directly compare our work with existing literature. So, we decided to draw comparisons with human fitness studies instead, even though the two areas were different. In equine physiology, studies on \gls{lac} were much scarcer than in human research. This gap gave us a chance to mix insights about the body with data about movement, adding depth to our research \ref{chapter:fatigue}. 

Logistical challenges were also encountered during the research. The collection of necessary data proved to be demanding, primarily due to our non-expertise in equine matters, which made it difficult to locate or convince participants. To surmount these challenges, collaboration was initiated with veterinarians and physiologists possessing robust connections within the equine community. This underlines the significance of building collaborative relationships in research, a practice that can yield mutual benefits for all parties involved.

Moreover, the limited availability of sensor data and the lack of familiarity among equine science researchers with computer science and data-sharing protocols within the computer science community posed significant obstacles to the public accessibility of data, rendering data sharing a challenging effort. It is worth noting that these issues may progressively ameliorate with time and as the volume of collected data increases. Additionally, as the field of equine sciences incorporates more elements of computer science and embraces data-sharing practices as routine, such challenges may become less pronounced.

\item{\textit{The value of user feedback becomes more evident in its absence}}

Studying fatigue in sport horses is especially challenging because these animals cannot provide direct feedback. Unlike humans who can communicate their sensations, horses cannot express how they feel in words. This creates a gap in our understanding, as the usual human-centric feedback methods do not apply. Without verbal communication, interpreting the equine experience becomes a complex task. It is needed to read their physical cues like sounds and facial expressions, much like decoding a unique language. This complexity highlights the value of user feedback, not just in advancing our understanding of equine fitness but also as a testament to the invaluable role of user-driven insights in scientific research.

\item{\textit{Collecting data from more subjects sometimes outweighs using more sensors}} 

In the context of data collection for research and analysis, the emphasis on diversity sometimes supersedes the pursuit of complexity. An exemplary illustration of this principle lies in the strategy of employing fewer sensors across a broader spectrum of horses, rather than inundating the data collection process with an excess of sensors. This pragmatic approach, as demonstrated in the study, has yielded compelling results. By widening the scope of data collection to encompass a more extensive range of horses, the research has generated insights that surpass the confines of specialization. Given the challenges associated with recruiting volunteers for data collection, or the limited availability of volunteers, a prudent approach involves allocating measurement resources to encompass a broader group rather than expending all resources on a select few. 

Notably, the model originally tailored to eventing horses has showcased its versatility by extending its utility to endurance horses (\ref{chapter:fatigue}). This capacity for cross-disciplinary applicability underscores the merits of prioritizing subject diversity in data collection. Such an approach leads to the development of models that exhibit adaptability and wider relevance. In essence, the decision to concentrate on broadening the spectrum of subjects has, in this case, engendered more insightful and practical research outcomes, exemplifying the advantages of variety over sheer sensor quantity.

\item{\textit{Data annotation or video recording during data collection is less effort than piecing events together later}} 

Whether in scientific investigations or other domains, annotating data during collection or recording relevant events can significantly affect the subsequent analyses. This proactive approach minimizes the need to painstakingly assemble fragmented data or decipher events from raw sources later in the process, ultimately saving valuable time and resources. This approach not only streamlines data handling for the data collector but also significantly benefits future users. 

The repercussions of unannotated data became evident in Chapters \ref{chapter:Speed} and \ref{chapter:Step}, where the data was obtained from external sources. In these instances, deciphering events and ensuring their accuracy was a time-consuming endeavor. Subsequently in our research journey, and having recognized the value of data annotation, significant improvements were implemented in the data collection process, notably in the \gls{set} measurement (Chapter \ref{chapter:data}). We efficiently utilized a pre-made template that specified the assignment of \gls{imu}s to horses and their respective starting times. Although this template greatly enhanced organization, it did not include timestamping of specific events. While video recording could have been advantageous for event timestamping and overall efficiency, it necessitated a dedicated team to simultaneously monitor horses undergoing the \gls{set} protocol. This experience underscores the significance of meticulous planning and a well-structured data collection strategy, not only simplifying immediate data management but also enriching the data's long-term utility for subsequent research and analysis.

\item{\textit{Prioritizing practicality over accuracy can be advantageous}} 

In the quest for scientific knowledge, it can be prudent to prioritize practicality over performance and accuracy. The trajectory of this prioritization has been evident in the course of the thesis, real-world conditions often challenge the idea of achieving theoretical perfection. This thesis highlighted the significance of aligning research methodologies with pragmatic considerations. For instance, the thesis encountered a practical scenario where employing just a single sensor for fitness assessment came at the expense of model accuracy. However, this trade-off proved to be a wise decision, as it rendered the process towards a more practical and user-friendly application. In another example, many studies have focused on fitness assessment in controlled laboratory settings. However, actual races and their exercises take place outdoors, in the field. Therefore, it is crucial to investigate the problem in conditions influenced by real-world factors. 

The critical lesson learned is that, sometimes, it is better to prioritize real-world practicality and user-friendliness over endlessly trying to improve absolute precision. This represents a significant shift in how we approach scientific research, emphasizing the importance of practicality alongside precision.

\item{\textit{Data collected for one purpose may hold research potential for another}}

This lesson highlights the flexibility of data, often exceeding its original purpose and becoming an asset in diverse research areas. We can observe this phenomenon in our research. Initially focused on assessing equine fatigue parameters (\ref{chapter:prepost} and \ref{chapter:fatigue}), our data collection revealed an opportunity to investigate the rider's impact on horse performance and its implications for fatigue levels (\ref{chapter:rider}). Similarly, datasets originally intended for genetic and t\"{o}lt parameter studies of Icelandic and French Montagnes horses were used for estimating the speed (\ref{chapter:Speed}). This showcases the transformative potential of data, adapting to unveil new research horizons beyond its initial objectives.

\end{itemize}





