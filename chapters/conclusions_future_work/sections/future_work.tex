\section{Outlook}
\label{sec:outlook}

In this section, we present prospective ideas that can be addressed to extend our work in future work.

\begin{itemize}[label={$\bullet$}, leftmargin=*]

\item{\textit{Real-time Assessment}}

All the models in this thesis operate offline, requiring data collection and subsequent feeding of data into models running on a laptop. Within the scope of this thesis, attempts have been made to assess model accuracy under conditions of limited resource consumption. These endeavors were directed towards the prospect of embedding the estimation model directly into the data-collecting \gls{imu}. However, it's important to note that the conversion of the current models to real-time capabilities was not pursued in this thesis.

Implementing real-time capabilities within the multimodal system (see Figure \ref{fig:wholepic}) would offer significant advantages. It would empower riders and trainers to make immediate decisions during exercise, providing a more accurate assessment of their activity and aiding in fatigue prevention. Similar approaches can be seen in human fatigue studies \cite{jiang_2022}. Real-time capabilities can also provide immediate access to speed information, particularly in situations where \gls{gps} signals are unavailable. Furthermore, real-time functionality can deliver instant updates on stride-related parameters like stride duration and biomechanical parameters, including pelvis roll. These features offer riders and trainers a real-time overview of important fitness metrics tailored to their specific training needs.

\item{\textit{Personalized Exercise Coaching}}

In the domain of sport horse training, each rider holds a distinct set of personal training objectives for their equine partner. These objectives may encompass a desire for enhanced speed or increased endurance, thereby lending significant relevance to the feedback derived from our multimodal system. It is noteworthy that riders often exercise discretion when gauging their horse's fatigue levels, as they may opt to curtail the training session before reaching high levels of fatigue to minimize risk. Conversely, for the purpose of augmenting fitness, riders may elect to persevere during phases of elevated \gls{lac}. 

In an alternate scenario, the aim may be to prolong the duration of reaching fatigue levels, necessitating the adoption of more efficient training strategies. For instance, maintaining a trot at a specific speed as opposed to cantering or modifying stride parameters by decreasing length and increasing duration could serve as a solution. Understanding the underlying training goals is pivotal, as it empowers real-time feedback mechanisms to offer more tailored and intelligent guidance to riders and trainers as they work towards their individualized objectives.


\item{\textit{Long-term Fitness Assessment}} % 

An essential outlook of this thesis lies in the long-term monitoring and assessment. The rationale behind this direction is to modify the models developed in this study to consider intra-individual fitness improvement over an extended period, which necessitates comprehensive long-term monitoring and consistent data collection from training sessions. This endeavor will involve designing a robust data collection plan, encompassing continuous monitoring and periodic assessments, to capture the evolution of key fitness parameters over time. 

The dynamic nature of horse performance and well-being can be effectively captured by advanced statistical analysis and the active identification of trends, patterns, and changes within the data, which must subsequently be incorporated into the fitness models. This long-term assessment not only has theoretical significance but also practical implications for refining training strategies, preventing injuries, and enhancing overall horse management in the equestrian industry. While challenges and limitations, such as data integrity and horse compliance, may be encountered, the potential for collaboration and data sharing with other researchers should be acknowledged to expand the scope of this endeavor and contribute to the collective knowledge in equine fitness assessment.
\end{itemize}