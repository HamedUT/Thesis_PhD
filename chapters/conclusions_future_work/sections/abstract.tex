\begin{quote}


In this thesis, we investigated how we can reliably and accurately measure the fitness of sport horses in a non-invasive manner. We hypothesized that we could enhance the performance and fitness assessment of sport horses by processing wearable sensor data through the development and implementation of intelligent systems and models. 

The content of this thesis is organized into two connected parts that assess this hypothesis through the integration of sensors and machine learning characterized by their increasing capabilities. In Section \ref{sec:intro_research_objective}, we formulated more specific research questions for each part, each directly aligned with the main research question. The corresponding sub-questions for each part are addressed in the subsequent five sections.

Within this chapter, we substantiate and affirm the validity of our hypotheses. Specifically, we provide evidence that crucial biomechanical and physiological parameters can be reliably estimated from inertial sensors through the utilization of machine learning techniques, even under resource constraints. 

Section \ref{sec:lessons_learned} offers a compilation of the lessons learned during the course of this research, while Section \ref{sec:outlook} delineates issues, suggests avenues for further investigation, and outlines directions for future work.

\end{quote}