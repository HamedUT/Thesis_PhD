\section{Introduction}
\label{sec:intro_introduction}

The use of wearable sensors has seen significant growth in recent years across various applications. These applications encompass health status monitoring \cite{jourdan_2021_the}, performance evaluation in athletes \cite{rana_2021_wearable}, and the development of activity recognition models \cite{ramanujam_2021_human}. Various tools are available for analyzing the data generated by these sensors, with machine learning particularly noteworthy for its capacity to automate the analysis process.

The process of developing machine learning models necessitates the labeling of collected data. This labeling must align with the specific measurement objective. Data labeling, in essence, involves enhancing sensor-generated data by appending informative descriptors, known as labels, to provide context. This enables a machine learning model to assimilate and understand the data in accordance with the assigned labels. To prepare a new model for labeling tasks, the data must first undergo labeling. Once trained, the model is capable of labeling new datasets. Developing models for data labeling not only brings economic advantages but also facilitates the automation of wearable-based applications.

Wearable sensors serve various purposes in the development of data labeling models, such as real-time labeling of sensor data related to wearer activities \cite{articllstm}. Subsequently, these developed models can find utility in real-world applications demanding automatic and real-time activity recognition.  

In studies focused on equine gait, it is crucial to determine whether a horse is under saddle or not, as the kinematics of a ridden horse differ from those of a horse moving on its own or being led by hand. Additionally, discerning the ridden status is vital for comprehending the impact of riders on horse health and performance. For instance, alterations in stride duration, stance, and swing phases (for the definitions, see Table \ref{tab:stride_related_parameters}) and changes in speed can serve as indicators of improved athletic performance or fatigue \cite{parkes_2019_the,Takahashi2021EffectsRaces}.

Furthermore, gait biomechanical parameters, such as body joint angles and displacements during a stride, are critical metrics for evaluating athletic performance and detecting lameness during movement \cite{Hardeman2019}. For example, parameters like MaxDiff and MinDiff play a pivotal role in assessing movement symmetry, which is essential for evaluating lameness \cite{Kramer2004}. 

Similar to other procedures, manual data labeling for ridden status can be accomplished through direct observation of the measurements. However, when it comes to automatic and real-time monitoring using wearable sensors, continuous observation of measurements is neither practical nor feasible. Consequently, our approach involves the development of classification models using data generated by inertial sensors positioned on the horse's body during both walking and trotting. These models aim to determine whether the horse is ridden or not-ridden.

We chose to study both walking and trotting since they exhibit distinct motion patterns, as outlined in Table \ref{tab:gait}. Therefore, it helps in the creation of versatile classification models that could effectively distinguish the ridden status of the horse based on different motion patterns. 

Our objective was threefold: first, to enhance the performance of these classification models. Second, to optimize the selection of body-mounted sensors employed in the process. Finally, we conducted an investigation into the parameters significantly influenced by the presence of a rider.

