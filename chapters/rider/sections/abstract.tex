\begin{quote}

In recent years, wearable sensors play an essential role not only in human gait analysis but also in equine gait studies. One of their roles is evaluating the rider’s effects on horse biomechanics, which affects the horse's well-being and performance. Therefore, it is necessary to identify whether the horse is being ridden or not before performing gait analysis. Despite numerous studies having developed machine learning models for labeling various gait characteristics, there exists a notable research gap in the automatic identification of the horse's ridden or unridden state. Particularly for real-time applications, it becomes essential to devise an optimized model that conserves computational resources. While several attempts have been made to create machine learning models for automating gait characteristic assessments, the automatic identification of the horse's ridden state has been overlooked. This chapter investigated the possibility of classifying horse locomotion to ridden/not ridden labels using minimal inertial sensors on the horse's body by identifying horse biomechanical parameters that get affected by rider. We presented a classification model that accurately labels the horse's ridden state using a minimum number of wearable sensors. Furthermore, the classification model was optimized in terms of required execution time. In conclusion, this chapter contributed a time-optimized classification model for the accurate identification of the horse's ridden state relying only on one wearable inertial sensor.

\end{quote}

\blfootnote{This chapter is a revised and extended version of the work originally published as \textit{'A machine learning approach to analyze rider’s effects
on horse gait using on-body inertial sensors'} in the Proceedings of The 20th International Conference
on Pervasive Computing and Communications (PerCom 2022).}