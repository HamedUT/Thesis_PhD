\section{Discussion}
\label{sec:discussion}

The random forest model trained on Sacrum+Limb \gls{imu}s features yielded the highest accuracy, sensitivity, and specificity, which indicates the importance of combining the extracted features from both \gls{imu}s to achieve a more accurate classification model. In addition, the model performance dependency on both \gls{imu}s shows that the rider affects limb and sacrum kinematics. 

On the models based on a single \gls{imu}, the limb-based model performed better than the sacrum-based model. From a practical aspect, equipping the limb with an \gls{imu} is more straightforward than mounting an \gls{imu} on top of the sacrum. The limb \gls{imu} can be put in a pocket, where the pocket is wrapped around the limb or a boot using Velcro straps. Wrapping the pocket around the \gls{imu} for an average user can be done with simple guidance. Mounting an \gls{imu} sacrum is usually done by sticking a double adhesive tape. However, the adhesive tape gets loosened due to the horse's sweat, and the sensor falls off. Moreover, finding the anatomical location of the sacrum requires equine anatomical knowledge. Therefore, the practical advantage of using limb \gls{imu} for classification adds up to its better performance result compared to sacrum \gls{imu}.

According to the outcome of \gls{nca} (Table \ref{tab:rider_features_results}), we selected the first ten features with the highest weights. The sacrum+limb \gls{imu}s model is highly dependent on these features for classifying the ridden or not-ridden status. In the following, we discussed the importance of these features in equine health and performance evaluation.

The magnitude of \gls{fft} first coefficient from the sacrum vertical acceleration signal was selected as the first-ranked feature by \gls{nca}. Implementing \gls{fft} analysis on a signal helps to determine the main harmonics of locomotion \cite{hayati_2019_analysis}. For instance, stride frequency can be calculated by deploying \gls{fft} on sacrum vertical acceleration signal \cite{Barrey1999MethodsHorses}. According to the left chart in Figure \ref{fig:rider_six}, horses presented a wider range of \gls{fft} magnitude values when being ridden. This can be explained by the excitement of horses while being ridden \cite{hall_2013_assessment}. Furthermore, swing duration was also selected as a top-ranked feature (the ninth feature), indicating the time duration of the limb in the air is affected by the presence of a rider. Therefore, it can be concluded that the rider affects the stride frequency and swing duration of the horse. This conclusion can also be seen in \cite{deuel_1988_effects}, where they showed the rider's urge on the horse increases the stride frequency. 

In addition to the first-ranked feature, the seventh and eighth-ranked features were also derived from the sacrum vertical acceleration signal. Sacrum vertical range of motion indicates the vertical displacement of the sacrum per stride. The importance of this parameter might be relevant to the rider and the additional weight that can limit its value.

From second to fifth-ranked features, all the features were based on the limb angular motions. This increase in limb angular range of motion (middle chart on Figure \ref{fig:rider_six}) might be due to the more excitement or discipline and rhythm in locomotion while ridden. Since the average stride frequency derived from the magnitude of \gls{fft} stayed the same between the two classes, the angular velocity of the limb decreases due to the higher angular range of motion (the right chart in Figure \ref{fig:rider_six})

Sacrum angular velocity around the longitudinal axis (sixth-ranked feature) can be related to the pelvis roll. Pelvis roll is one of the important parameters for examining lameness and demonstrates the symmetry of locomotion \cite{Pfau2013}. Therefore, the rider might affect the pelvis roll value compared to the time that the horse is roaming freely.
