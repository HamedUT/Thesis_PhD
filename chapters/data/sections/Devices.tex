\section{Measuring Devices} \label{datacollection-devices}

Equine gait analysis relies on the integration of advanced measurement devices to facilitate the collection of data on equine locomotion. These devices allow researchers to capture and analyze various parameters associated with gait mechanics, such as joint angles, stride characteristics, forces exerted, and temporal patterns. The data collection process involves placing sensors or markers on specific anatomical landmarks of the horse's body to ensure accurate and reliable measurements. Careful attention is paid to calibration procedures and sensor placement to minimize potential sources of error and ensure data integrity. 

The use of measurement devices in equine gait analysis provides valuable insights into the biomechanics, performance, and overall well-being of horses. As a result, the collected data serves as a foundation for evidence-based training methods, performance enhancement, injury prevention, and the optimization of equine athletic capabilities. This section introduces several commonly utilized types of measurement devices in equine gait analysis, each followed by their advantages and disadvantages:
\begin{enumerate}
\item \textbf{\gls{omc}:} \gls{omc} utilizes multiple high-speed cameras to track the position of reflective markers placed on specific anatomical landmarks of the subject's body. By precisely measuring the three-dimensional movement of these markers, \gls{omc} provides valuable data on joint angles, stride length, and spatiotemporal parameters \cite{higginson}. This technology enables researchers to gain a detailed understanding of gait biomechanics.
\vspace{1cm}
\begin{itemize}
\item[] \textbf{\small Advantages:}
\begin{itemize}
\item Provides highly accurate and detailed 3D movement data
\item Allows for precise tracking of markers on the horse's body
\end{itemize}
\item[] \textbf{\small Disadvantages:}
\begin{itemize}
\item Requires specialized equipment, setup, and indoor laboratory environment; not suitable for outdoor exercise \cite{darbandi_2021_using}
\item Marker occlusion may occur, affecting data accuracy \cite{occulusion}
\end{itemize}
\end{itemize}

\item \textbf{Force Plates:} Force plates, embedded in the ground, measure the forces exerted by the horse's hooves during different phases of the gait cycle. These plates allow for the evaluation of ground reaction forces, providing insights into weight distribution, asymmetries, and potential abnormalities. By assessing force plate data, researchers can gain valuable information on the impact and loading patterns experienced by the horse's limbs during movement.
\begin{itemize}
\item[] \textbf{\small Advantages:}
\begin{itemize}
\item Measures temporal data on forces exerted by the horse's limbs with high accuracy during the stance phase
\end{itemize}
\item[] \textbf{\small Disadvantages:}
\begin{itemize}
\item Requires the horse to be trained to perform on the force plate
\item Intended for stationary use only; not suitable for exercise on the field
\end{itemize}
\end{itemize}

\item \textbf{Pressure Mats or Plates:} Pressure mats or plates are placed under the horse's hooves and measure the distribution of pressure during locomotion. By analyzing pressure data, researchers can gain insights into hoof loading patterns, potential imbalances, and the interaction between the hoof and ground surface. This information helps in assessing the functionality of the horse's limbs and aids in identifying potential issues related to weight-bearing and biomechanics.
\begin{itemize}
\item[] \textbf{\small Advantages:}
\begin{itemize}
\item Measures pressure distribution under the horse's hooves
\item Implementation in a customized horseshoe for temporal pressure distribution during exercise
\end{itemize}
\item[] \textbf{\small Disadvantages:}
\begin{itemize}
\item Requires proper calibration for accurate results
\item Sensitivity to environmental factors such as moisture and temperature \cite{pressure}
\item Individual customization is required for embedding the sensor in the horse's shoe.
\end{itemize}
\end{itemize}

\item \textbf{Inertial Measurement Unit (IMU):} \gls{imu}s are small, wearable sensors that can be attached to different body parts of the horse, such as hooves, limbs, sacrum, withers, and head. \gls{imu}s capture data on acceleration, orientation, and angular velocity, allowing for the monitoring of gait parameters during different activities. These portable devices offer advantages in terms of convenience and versatility, enabling the assessment of equine gait in both controlled environments and natural settings.
\begin{itemize}
\item[] \textbf{\small Advantages:}
\begin{itemize}
\item Portable and easy to use
\item Provides real-time data on movement and orientation
\end{itemize}
\item[] \textbf{\small Disadvantages:}
\begin{itemize}
\item Accuracy may vary depending on sensor quality
\item Requires sophisticated algorithms for data processing and interpretation
\end{itemize}
\end{itemize}

\item \textbf{\gls{emg}:} \gls{emg} is a measurement technique that assesses the electrical activity of muscles. In gait analysis, EMG sensors are placed on specific muscles to evaluate muscle activation and coordination during movement. By analyzing the patterns and timing of muscle activity, researchers can gain insights into muscular function, recruitment strategies, and potential imbalances. EMG data complements other measurements by providing valuable information about muscle contributions to equine locomotion.
\begin{itemize}
\item[] \textbf{\small Advantages:}
\begin{itemize}
\item Assess muscle activity and recruitment patterns
\end{itemize}
\item[] \textbf{\small Disadvantages:}
\begin{itemize}
\item Placement of sensors may be challenging in some cases
\item Limited to surface muscle activity measurement
\end{itemize}
\end{itemize}

\item \textbf{Treadmills:} Treadmills, although not measurement devices on their own, serve as valuable tools for sports measurements in laboratory environments. Equipped with a range of sensors and force-measuring devices, laboratory treadmills facilitate gait analysis in controlled settings. They enable standardized testing and repetitive measurements under controlled conditions, allowing researchers to simulate various gait patterns and intensities by adjusting speed and incline. Consequently, treadmills facilitate the precise measurement of parameters including stride length, duration, and symmetry. The controlled environment provided by treadmills ensures consistent and comparable data collection across different horses, thereby aiding in the evaluation and comparison of equine performance.
\begin{itemize}
\item[] \textbf{\small Advantages:}
\begin{itemize}
\item Controlled environment for consistent gait analysis
\item Ability to adjust speed and incline for specific protocols
\end{itemize}
\item[] \textbf{\small Disadvantages:}
\begin{itemize}
\item Intended for stationary use only; may not fully replicate real-world field conditions
\item Requires specialized equipment and space
\end{itemize}
\end{itemize}
\end{enumerate}

In addition to the measurement devices, equine gait analysis may also involve the assessment of various physiological parameters to gain a more comprehensive understanding of a horse's performance and well-being. These physiological parameters can provide insights into the cardiovascular, respiratory, and metabolic responses during exercise or movement. Some of the physiological measurements commonly used in equine gait analysis (including advantages and disadvantages for each) are as follows:

\begin{enumerate}
\item \textbf{\gls{hr} Monitors:} \gls{hr} monitors are commonly used in equine gait analysis to assess cardiovascular responses during exercise. These devices measure the horse's \gls{hr}, providing valuable information about exercise intensity, workload, and recovery. By monitoring \gls{hr} changes throughout different gait phases and exercise protocols, researchers can gain insights into the cardiovascular demands placed on the horse and assess their fitness levels.
\begin{itemize}
\item[] \textbf{\small Advantages:}
\begin{itemize}
\item Provides real-time monitoring of the horse's \gls{hr}
\item Helps assess cardiovascular response to exercise
\end{itemize}
\item[] \textbf{\small Disadvantages:}
\begin{itemize}
\item External factors like sweat and intensive movements can interfere with accurate readings
\end{itemize}
\end{itemize}

\item \textbf{Blood/Plasma \gls{lac} Analysis:} Blood/plasma \gls{lac} analysis involves the measurement of lactate levels in the horse's blood/plasma. This parameter serves as an indicator of anaerobic metabolism and provides insights into exercise intensity and muscle fatigue. By analyzing \gls{lac} during and after exercise, researchers can evaluate the horse's energy metabolism, assess their aerobic and anaerobic capacities, and make informed decisions regarding training intensity and recovery strategies.
\begin{itemize}
\item[] \textbf{\small Advantages:}
\begin{itemize}
\item Measures \gls{lac} in blood or plasma as an indicator of muscle fatigue
\item Provides quantitative data on anaerobic metabolism
\end{itemize}
\item[] \textbf{\small Disadvantages:}
\begin{itemize}
\item Requires blood extraction from the jugular vein \cite{munsters_2014_exercise}, which can be invasive and stressful for the horse \cite{JANSEN200938}
\end{itemize}
\end{itemize}

\item \textbf{Oxygen Consumption (VO$_2$ max) Analysis:} VO$_2$ max analysis is a technique used to measure the amount of oxygen a horse consumes during exercise. By evaluating VO$_2$ max, researchers can assess the horse's aerobic capacity, energy metabolism, and efficiency. This information helps in determining the horse's fitness level and identifying areas for improvement. VO$_2$ max analysis provides valuable insights into the horse's physiological responses to exercise and aids in designing effective training programs.
\vspace{1cm}
\begin{itemize}
\item[] \textbf{\small Advantages:}
\begin{itemize}
\item Measures the horse's oxygen consumption during exercise, reflecting its aerobic capacity
\item Provides objective data on the horse's fitness level and metabolic efficiency
\end{itemize}
\item[] \textbf{\small Disadvantages:}
\begin{itemize}
\item Requires specialized equipment and expertise for accurate measurement
\item Can be challenging to perform VO2 analysis during high-intensity exercise
\item May involve the use of a mask or apparatus that could potentially affect the horse's natural breathing pattern
\end{itemize}
\end{itemize}

\item \textbf{Respiratory Gas Analysis:} The respiratory gas analysis involves the measurement of the horse's breath-by-breath gas exchange during exercise. By analyzing the composition of inhaled and exhaled gases, researchers can assess the horse's oxygen uptake, carbon dioxide production, and respiratory exchange ratio. This information provides insights into the horse's aerobic capacity, energy metabolism, and efficiency. The respiratory gas analysis aids in determining the horse's fitness level, identifying potential respiratory limitations, and optimizing training programs to enhance performance.
\begin{itemize}
\item[] \textbf{\small Advantages:}
\begin{itemize}
\item Assesses respiratory parameters like oxygen consumption and carbon dioxide production
\item Provides insights into aerobic capacity and energy expenditure
\end{itemize}
\item[] \textbf{\small Disadvantages:}
\begin{itemize}
\item Requires specialized equipment and expertise for accurate measurements
\item Can be challenging to use during high-intensity exercise
\end{itemize}
\end{itemize}

\item \textbf{Thermography:} Thermography is a non-invasive technique used to measure surface temperature variations in the horse's body. By capturing infrared images, thermography can detect and map areas of increased or decreased temperature, indicating changes in blood flow, inflammation, or potential injuries. In equine gait analysis, thermography provides insights into the thermal patterns associated with different gait phases, highlighting areas of potential stress or abnormalities. This information assists in identifying early signs of musculoskeletal issues and aids in the assessment of the horse's biomechanical balance and overall health.
\begin{itemize}
\item[] \textbf{\small Advantages:}
\begin{itemize}
\item Measures skin surface temperature to identify areas of inflammation or injury
\item Non-invasive and provides a visual representation of temperature distribution
\end{itemize}
\item[] \textbf{\small Disadvantages:}
\begin{itemize}
\item Interpretation of thermographic images requires expertise
\item External factors like environmental temperature can affect readings
\item Can be challenging to use during exercise in real-world settings
\end{itemize}
\end{itemize}
\end{enumerate}

In summary, after examining the measurement devices for both biomechanical and physiological analysis in equine gait, \gls{imu} and \gls{hr} monitors emerge as the most suitable choices for the study. The selection of these devices aligns closely with the study objectives, which aim to evaluate the fitness levels of sport horses and collect data during exercise sessions in real-world settings, specifically field training. 

The \gls{imu} offers several advantages, including its portability and versatility, allowing for the collection of accurate biomechanical data on joint angles and segmental motion. Moreover, \gls{hr} monitors provide real-time monitoring of the horse's \gls{hr}, enabling the assessment of cardiovascular response and fitness levels. Additionally, both the \gls{imu} and \gls{hr} monitors satisfy the requirement for non-invasive data collection, ensuring minimal interference with the horse's natural movement and reducing potential stress. By employing the \gls{imu} for biomechanical measurements and \gls{hr} monitors for physiological data, this research aims to assess and understand the fitness, and therefore, performance and well-being of sport horses in real-world environments.

