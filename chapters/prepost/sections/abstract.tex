\begin{quote}

Detection of fatigue helps prevent injuries and optimize the performance of horses. Previous studies tried to determine fatigue using physiological parameters. However, measuring the physiological features, such as collecting blood for \gls{lac} measurement, is invasive and can be influenced by different factors. In addition, the measurement cannot be done automatically and requires an expert or a veterinarian for sample collection. This study investigated the possibility of automatically detecting fatigue non-invasively using a minimum number of body-mounted \gls{imu}. In this study, sixty sport horses were measured using \gls{imu}s mounted on the body before and after high and low-intensity exercises during walk and trot. From the \gls{imu} signals, biomechanical-related features were calculated per stride. Using machine learning methods, we selected the features with the highest weights as important fatigue indicators, and based on them, we developed classification models to assign a non-fatigue or fatigue label to the strides. This study confirmed that a number of features can indicate fatigue in horses. The most important fatigue indicators at walk, trot, and both gaits were stance duration, swing duration, and limbs range of motions, respectively. The fatigue models resulted in high accuracy during both walk and trot. In conclusion, fatigue can be classified non-invasively during exercise by equipping horses with \gls{imu}s.

\end{quote}

\blfootnote{This chapter is a revised and extended version of the work originally published as \textit{'Detecting fatigue of sport horses with biomechanical gait features using inertial sensors.'} in PLOS ONE Journal in 2023.}