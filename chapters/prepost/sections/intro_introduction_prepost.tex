\section{Introduction}
\label{sec:intro_introduction}

The equestrian sport is under progressive scrutiny of public opinion regarding equine welfare, which makes it essential to provide more insight and transparency into their physical and biomechanical demands. In this context, fatigue emerges as a critical factor in performance and welfare. Horses engaged in exercise and competition typically experience some degree of fatigue due to exertion. Excessive or prolonged fatigue during these activities can lead to overtraining and injuries \cite{graaf232}. By assigning fatigue as an indicator \cite{Takahashi2018ElectromyographicExercise},  it becomes possible to proactively prevent these injuries and overtraining issues. 

In contrast to human athletes, horses cannot express their fatigue levels, requiring riders and trainers to closely monitor the horse to assess their fatigue. The lack of proper qualitative determination may result in inadequate training stimulus or insufficient recovery periods. 

Finding a balance between exercise and recovery periods is a challenging yet essential aspect of maintaining optimal health and performance. Several studies showed that an abrupt increase in exercise load results in an increased risk of injury, as the body has not adapted well to the earlier exercise responses  \cite{Munsters2020AStudy}. In addition, fatigue has several consequences on the performance \cite{Arfuso2021PeripheralHorses,Ropka-Molik2019TheHorses,Rivero2001CoordinatedTraining.}, health and welfare of the horse \cite{Hogg2021SymbiosisSports}. In severe cases, fatigue can cause horses to collapse and result in sudden death during competitions \cite{verheyenarticle}. Therefore, monitoring fatigue during exercise and competition is vital for injury prevention, performance optimization, and welfare improvement. 

Understanding fatigue mechanisms and indicators can help prevent fatigue-related injuries \cite{Takahashi2018ElectromyographicExercise}. In many studies, fatigue is referred to as the moment that the horse "cannot maintain the pace on treadmill despite verbal encouragement" \cite{Colborne2001ElectromyographicStudy.,Warren1999TheFatigue,Takahashi2020DoHorses,Curtis2006ObservationsRacehorses,Kusano2007RelativeHorses,Bayly2019EffectExercise,wickler_2006_stride,articlefer,Bowers1999InfluenceRacehorses.,Savage2005EffectsThoroughbreds,Jose-Cunilleras2006CardiacRacehorses,Buhl2018EffectTrotters}. This indicator can be valuable in a treadmill measurement setting when accompanied by veterinarians to monitor the horse health status during measurement. However, it is qualitative and not practical for use during exercise or competition in the field \cite{munsters_2014_exercise}. In addition, the point at which an individual voluntarily halts the exercise varies inter-individually. Some may stop before the occurrence of fatigue, while others may push themselves well beyond their limits \cite{Baron511}. Furthermore, monitoring and analyzing the exercise in the field is more challenging than measurement on a treadmill since multiple factors change between measurements, which might be surface types, weather conditions, rider effects, and speed \cite{munsters_2014_exercise,darbandi_2022_a}. 

\subsection{Fatigue assessment methods}

One of the common methods for assessing horse fitness and consequently fatigue is performing a \gls{set}. In general, \gls{set} evaluates the physiological responses of horses to training. A \gls{set} that is performed on a field, often referred to as a field \gls{set}, should closely replicate the competition environment. Field \gls{set}s usually consist of multilevel incremental exercise steps during which \gls{lac}, \gls{hr}, and speed are measured. To yield more relevant results, the \gls{set} must be tailored to the specific discipline and competition level. Therefore, a discipline-specific component is integrated into the \gls{set}, encompassing technical skills relevant to that discipline. As a result, the intensity of a \gls{set}, determinable by \gls{hr} and LA, can vary significantly among different equestrian disciplines \cite{munsters_2014_exercise,ETO2004139,Piccione2010BloodHorses}.

Among physiological measurements during \gls{set}, the \gls{hr} can be evaluated by equipping the horse with a \gls{hr} sensor. However, measuring \gls{lac} is invasive, as it necessitates the extraction of a blood sample from the horse's veins. Additionally, the measurement is discrete, requiring multiple interruptions during exercise to collect these blood samples. Moreover, physiological parameters can be influenced by the horse's emotional state \cite{JANSEN200938}.

In addition to the physiological features, biomechanical features can also present fatigue changes. As an example from horse fatigue literature, stride duration increases and speed decreases \cite{Cottin2006EffectTraining,Colborne2001ElectromyographicStudy.,Williams2018ElectromyographyTechnology,Pugliese2020EffectStudy,Takahashi2021EffectsRaces}. Only a few biomechanical features were investigated in fatigue literature despite numerous features studied in performance-related literature. For instance, stride length, stride duration, and limb angular range of motion were studied and compared as performance indicators \cite{Barrey1999MethodsHorses,parkes_2019_the}. 

With \gls{imu}s, a wider range of biomechanical features can be monitored, even in real-time assessments. \gls{imu}s are capable of continuous measurement in contrast to the discrete measurement of \gls{lac}. They are compact, non-invasive, and can be simply mounted on the body. Thus, they ease the calculation of the gait biomechanics features by using their output signals \cite{456}. Thanks to the portability and measuring capabilities of \gls{imu}s in assessing gait biomechanics features, it becomes possible to investigate how these features vary in response to exercise and the potential fatigue.

\subsection{Approach}
This chapter outlines a three-step approach for assessing the fatigue of sport horses using biomechanical parameters. The first step is to identify the biomechanical features that are closely correlated to fatigue using machine learning algorithms. The next step is to detect fatigue using the identified features while minimizing the number of body-mounted \gls{imu}s to enhance the practicality of measurements. Finally, in order to understand the effect of training intensity on biomechanical parameters, comparisons of the features values between two levels of exercise intensity were conducted. This chapter takes these steps to investigate equine fatigue indicators and classify fatigue based on extracted biomechanical features from a minimum number of body-mounted \gls{imu}s.  