\section{Discussion}
\label{sec:discussion}

The primary objective of this chapter was to explore the potential of detecting fatigue through the analysis of kinematics parameters. The chapter examined key biomechanical indicators of fatigue and investigated how gait type and the intensity level of the \gls{set} influenced these indicators. Furthermore, the importance of selected indicators was determined through the application of machine learning techniques to the data, with an evaluation of their performance. 

This chapter marks the first instance in equine literature where machine learning methods have been employed to identify critical biomechanical indicators of fatigue. Additionally, no previous study has undertaken the development and testing of pre-\gls{set}/post-\gls{set} classification models for horses. This chapter successfully created highly accurate classification models using as few as three to six biomechanical features.

%\subsection{Effects of IMU locations on the model performance}
All the subsets included at least three limb features, which presented the limbs as important locations for mounting the \gls{imu} on the horse body, independent of gait type and \gls{set} intensity level. In fact, by attaching one \gls{imu} to a front limb, 86\% and 80\% accuracy can be achieved when the fatigue state is being identified during high-intensity \gls{set} trot and Walk+Trot. Moreover,  the inclusion of an additional \gls{imu} on the hind limb led to a notable improvement in accuracy, achieving an impressive 95\% accuracy rate in dataset 1 within the walking subset. The decrease in \gls{imu}s resulted from feature selection, facilitating the practicality of equipping \gls{imu} on the body. 

Features extracted from the poll \gls{imu} were not identified as important features by any of the models used in the analysis. This could be attributed to several factors. Firstly, when horses are introduced to a new environment, they often become distracted, looking around and familiarizing themselves with their surroundings. This distraction can lead to variations in head position that may not be adequately captured by the \gls{imu} sensors. Additionally, variations in the forces applied by different handlers during in-hand walk and trot can potentially disrupt the \gls{imu} signals, causing the extracted features to no longer accurately represent the normal head position of the horses during locomotion. 

%\subsection{Features selected as fatigue indicators}
The longitudinal displacement of front limbs was the only feature present in all subsets, representing itself as an important fatigue indicator for both intensity levels and gaits. This feature is aligned with horse locomotion and the step length vector. Therefore, it can be inferred that the longitudinal displacement of limbs can be correlated to the step length. 

According to Figure \ref{walktrot} and Figure \ref{ind_walktrot}, independent of \gls{set} intensity level, the front limb longitudinal displacement of 47 and 52 horses (more than 78 and 86 percent of all horses) became shorter after exercise during walk and trot, respectively. This result is aligned with the outcomes of previous studies \cite{Colborne2001ElectromyographicStudy.,Takahashi2021EffectsRaces}. A possible explanation for the decrease in longitudinal displacement could be attributed to a reduction in muscle stiffness in the limb as a result of fatigue \cite{wickler_2006_stride}. Furthermore, the difference in this feature between the pre-\gls{set} and post-\gls{set} periods for both low and high-intensity \gls{set}s is displayed in Figure \ref{prepostlong}. It's evident that in the case of high-intensity \gls{set}s during Walk+Trot, the length became shorter after the \gls{set}, whereas after low-intensity \gls{set}s, this shortening was not consistently observed in all cases. Based on the findings, it can be concluded that the performance of horses may not be significantly affected by less intense exercise, as indicated by the low levels of \gls{lac} and the absence of excessive fatigue in the limb muscles.


Regardless of \gls{set} intensity and during walk or trot, the protraction/retraction angle of hind limbs appeared as an important indicator, according to Table \ref{featuresss results}. Similar to the decrease in the length of longitudinal limb displacement, the protraction/retraction angles of hind limbs were also decreased (Figure \ref{walktrot}), which might be due to the lack of force in limb muscles caused by fatigue.

Stance duration and swing duration were specified as important fatigue indicators during walk and trot, respectively. Stance duration was increased in 54 out of 60 horses (90 percent) after \gls{set} during the walk (as illustrated in Figure \ref{ind_walktrot} and Figure \ref{walk}). Conversely, the swing duration decreased during the trot (as depicted in Figure \ref{ind_walktrot} and Figure \ref{trot}) in 43 out of 60 horses, which accounts for more than 71 percent of the total horses studied. 

Owing to the importance of gait events features, we also investigated the other related features that were not selected by the feature selection system. The duration of stride during Walk+Trot increased, consistent with previous findings reported in the literature \cite{Cottin2006EffectTraining,Colborne2001ElectromyographicStudy.,Williams2018ElectromyographyTechnology, Pugliese2020EffectStudy,parkes_2019_the}. From a biomechanical perspective, the increase in stride duration can be attributed to a reduction in the activity in the muscles responsible for generating propulsive force \cite{Takahashi2021EffectsRaces}. This extension of the stride allows the muscle to shorten at an optimal rate, thereby enabling them to produce sustained power and accumulate more work \cite{wickler_2006_stride,johnston}. Stance duration increased in the walk as well as trot, and swing duration was approximately the same pre-\gls{set} and post-\gls{set} during the walk, while it was decreased during the trot. It can be seen in Figure \ref{prepoststride} that the walking stance duration was longer, and the trotting swing duration was shorter in higher intensity \gls{set}. Combining the walking and trotting strides, stance, swing, or stride duration were not indicated as distinguishing fatigue indicators. Overall, it can be derived that the significant changes of gait events features are dependent on the gait type and independent from the \gls{set} intensity level.


%\subsection{Effects of the subsets on the model performance}
By extracting the few selected features (Table \ref{featuresss results}) from strides, these models accurately classify the stride to fatigue/non-fatigue. The classification model trained on the walk subset of dataset 1 used no upper body features, while the models based on trot and Walk+Trot subsets of dataset 2 (high-intensity \gls{set}) used only features extracted from front limbs. 

The classification accuracies of the models were also different. For example, if both stride gait type and \gls{set} intensity level are unknown, the classification accuracy would be 82\%. In addition, if only the \gls{set} intensity level is known, for the lower level, the model yields higher accuracy (83\%) than the higher level of intensity (80\%). Furthermore, if stride gait type is known, we can achieve high model performance for walk with 95\% accuracy during high-intensity \gls{set}, 100\% during low-intensity \gls{set}, and 95\% accuracy if the intensity level is not known. In addition to walking strides, the trotting classification models based on known \gls{set} levels (86\% in high intensity and 88\% in low intensity) suggest better results than the model's accuracy based on the mixture of high and low \gls{set} intensity levels (83\%). Overall, the performances of the models in all subsets are higher for low \gls{set} intensity than high \gls{set} intensity, which can be explained due to different subjects in high intensity and low intensity datasets.

%\subsection{Comparison to the state-of-the-art}
Since there was no study on the classification of equine fatigue/non-fatigue, we compared the results with two studies on human fatigue. In one study, the walking patterns of seventeen subjects were classified as fatigue/non-fatigue induced by a squatting exercise \cite{Zhang2014}. The accuracy of the classification model was 96\%, which was lower than the accuracy of low-intensity \gls{set} classification model (100\%) in the current study but higher than the accuracy of the model based on all \gls{set}s. In another study, fatigue was induced by manual material handling sessions on thirty participants \cite{Baghdadi2018}. The result of the walking fatigue/non-fatigue classifier was 90\%, which was lower than all the three walking models in this study. 

The mentioned studies were similar in data collection and analysis to the current study, in which they used \gls{imu} for data measurement and machine learning for data analysis. Therefore, the models reported in this chapter can potentially outperform the classifiers in the human fatigue literature with a comparable study basis.