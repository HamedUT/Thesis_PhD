\section{Conclusion}
\label{sec:conclusion}

This study demonstrated that mounting only one \gls{imu} on a front limb makes it possible to monitor the value changes of important biomechanical indicators of fatigue induced by exercise. We presented walking stance duration and trotting limb longitudinal displacement as two biomechanical fatigue indicators, where most horses tend to increase and decrease respectively when fatigued. In addition, by building machine learning models on biomechanical parameters as input features, fatigue can be detected at 95\% and 83\% accuracy during walk and trot, respectively.  Using the results of this study in practice can help researchers and equestrians improve the welfare of horses and enhance training sessions by preventing the horse from excessive fatigue. In next Chapter, the classification of horse fatigue levels using \gls{imu}s will be improved by evaluating and adding more levels of fatigue.
