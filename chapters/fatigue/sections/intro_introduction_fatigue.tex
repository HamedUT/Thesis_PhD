\section{Introduction}
\label{sec:fatigue_introduction}

Fatigue in horses refers to a state of reduced physical and physiological capacity as a result of prolonged or intense exercise. This state can significantly impact performance and the overall welfare of horses, potentially leading to adverse effects on multiple body systems, including muscular, cardiovascular, and respiratory functions \cite{Taylor2008-mp}. Incorporating fatigue management into equestrian sports is crucial for the well-being and optimal fitness of horses. By understanding the physical and biomechanical demands placed on these animals and actively monitoring and addressing fatigue, trainers and riders can contribute to the longevity, health, and success of equine athletes while upholding ethical standards and public opinion \cite{Arfuso2021PeripheralHorses,Rivero2001CoordinatedTraining.}. 

Human athletes often express their fatigue level based on a \gls{rpe}, which can be used as reliable feedback for managing the exercise intensity and assessing fatigue \cite{carrie}. In contrast, horses cannot verbally express their fatigue state. Therefore, another metric is required for monitoring their fatigue level. Since fatigue is a multidimensional phenomenon, there is no single definition that is universally accepted across equine studies. In many studies, fatigue has been defined as the point at which a horse is unable to maintain its pace on a treadmill despite verbal encouragement \cite{Bowers1999InfluenceRacehorses.}. This definition is practical in a controlled setting, but it is not feasible for on-field exercise or competition as other factors like environment and rider influence also come into play.

Traditionally, blood \gls{lac} measurement during exercise has been widely used for fatigue evaluation in human athletes \cite{Theofilidis2018-yz,Ishii2013-mp}. \gls{hr}, while informative, can be influenced by various factors. \gls{lac} serves as a more direct indicator of metabolic stress and anaerobic activity, offering insights into muscle fatigue that \gls{hr} alone may not always capture effectively, especially not when used under field circumstances and ridden by a rider. However, the measuring procedure for sport horses poses significant challenges; it involves an invasive procedure, requiring the multiple extraction of blood samples from the jugular vein. Aside from causing discomfort and stress to the animal, repeated blood samplings during the training sessions or exercise tests are impractical and can disrupt the natural training or testing routine, as mentioned in Chapter \ref{chapter:prepost}. 

To address the challenges associated with fatigue assessment in sport horses, innovative technologies have emerged as valuable tools in recent years. Consequently,  there is a need for a non-invasive method that can replace \gls{lac} measurement, ensuring a thorough and dependable evaluation of fatigue in sport horses. In this study, we propose the utilization of \gls{imu} and \gls{hr} monitors as a solution for accurately estimating \gls{lac} in sport horses. \gls{hr} monitors in horses are used during training and exercise for tracking cardiovascular responses, optimizing workout intensity, and evaluating recovery \cite{munsters_2014_exercise}. Moreover, \gls{imu} sensors offer the ability to capture the movement with high precision. They have been designed for continuous measurement as opposed to the discrete measurement of \gls{lac}. They are small, non-invasive, and easily mountable on the body. By analyzing their output signals, biomechanical parameters can be calculated and then monitored. By monitoring key biomechanical parameters, we will be able to measure horse's fatigue levels during training sessions. 


The utilization of machine learning techniques is becoming progressively more prominent in the field of equine studies \cite{Alexeenko}. To estimate and classify different aspects of horse health and performance, studies have used various features for training their machine learning models. One study focused on biomechanical parameters to implement a binary classification of fatigue \cite{munsters_2014_exercise}. The most significant features from that study were stride duration, stance duration, stride length, speed, and limb protraction/retraction range of motion. The first two increased, while the latter three decreased due to fatigue. A number of studies have also used signal-based features for modeling, including time- and frequency-based features from chapters \ref{chapter:Speed} and \ref{chapter:rider}. Overall, both types of features convey significant information and can help in the development of models.

Many studies on human fatigue estimation used \gls{rpe} as the fatigue indicator. Aguirre et al. \cite{Aguirre2021-ab} used the data from ambulatory sensors to classify the fatigue from sit-to-stand exercise into three states, low, moderate, and high \gls{rpe}, which presented an 82.5\% accuracy. Op De Beéck et al. \cite{OpDeBeeck2018pq} estimated \gls{rpe} of runners using the data from four body-mounted \gls{imu}s and machine learning. The study also compared the accuracy between the two models based on all and individual participant(s), which indicated better results for the latter. Jiang et al. \cite{Jiang2021-cd} trained a model for fatigue prediction and quantification based on the data from \gls{imu} using random forest and \gls{cnn}. Their subject-dependent regression model showed a high correlation for continuous fatigue detection. Nevertheless, the absence of a metric comparable to \gls{rpe} for horses, makes it unfeasible to consider it for the assessment of exercise intensity and fatigue.

In this study, we compared the performance of different machine learning algorithms for modeling fatigue using motion and \gls{hr} data. To overcome the absence of \gls{rpe}, we considered \gls{lac} as the fatigue indicator. Furthermore, we examined the influence of the \gls{imu} placement on the horse's body on the performance of the estimation model. This approach holds the potential to assist in the practicality of fatigue assessment in sport horses during training, providing trainers and veterinarians with actionable data for informed decision-making and improved health outcomes. 
