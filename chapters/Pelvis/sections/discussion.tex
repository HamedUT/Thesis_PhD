\section{Discussion}
\label{sec:discussion}

The Bland-Altman analysis results demonstrated high agreements between the measurement devices within the methods. Therefore, it becomes evident that only one \gls{imu} is sufficient for a reliable and accurate calculation of the pelvis roll angle, irrespective of the chosen method. This simplifies the data collection process and minimizes the equipment requirements, thereby enhancing the practicality and accessibility of the measurement process. In addition, this outcome underlines the versatility and robustness of \gls{imu} in the assessment of equine biomechanics, eliminating the necessity for \gls{omc} in special laboratory settings.







The differences observed between methods were notably more pronounced than the differences within methods and between the measuring devices. This discrepancy can be attributed to variations in the respective centers of rotation for each method. Specifically, for the roll method, the center of rotation was situated directly above the sacrum, while for the inverse sine \gls{imu} method, it was positioned vertically below the sacrum. Consequently, the roll method yields a distinct angle in contrast to the inverse sine approach. Achieving precise pelvis roll angle measurements necessitates the identification of the exact pelvis center of rotation. However, this task becomes intricate due to the dynamic nature of locomotion, causing the center of rotation to fluctuate vertically throughout the movement.



Nonetheless, it is crucial to acknowledge that the angles measured did not exhibit an exact match with the actual pelvis roll angle, as we discussed earlier. While employing a uniform methodology throughout all these studies may indeed establish a strong and reliable reference point, especially when investigating alterations in the pelvis roll angle, it becomes particularly valuable for continuous, long-term performance monitoring. 

By adhering to a consistent metric and similar parameters, we suggest the measurement of hip hike as an alternative approach. Hip hike has been widely used in clinical settings, especially for assessing lameness, making it a well-established and proven metric for evaluating equine biomechanics and gait patterns. The adoption of this metric offers both standardization of measurement and the advantage of compatibility with existing clinical assessment protocols in equine performance evaluation \cite{Greve2018, Day2020}. 