\section{Conclusion}
\label{sec:conclusion}

The validation of the pelvis roll angle calculation method not only strengthens the credibility of employing an \gls{imu} as a valuable tool for evaluating angles associated with body joint movements but also underscores the broader applicability of this methodology. Utilizing a single \gls{imu} in conjunction with the roll method, which integrates the angular velocity signal recorded by the \gls{imu} over time, we have established the consistent and accurate calculation of the pelvis roll angle.

This level of reliability extends to other sites within the equine body, including the withers and limbs. When an \gls{imu} is correctly mounted on these joints and the angular velocity signal is appropriately integrated, we can confidently calculate joint angles. This not only signifies the versatility and robustness of the \gls{imu} approach but also opens up a world of possibilities for accurately assessing and monitoring various aspects of equine biomechanics and movement.