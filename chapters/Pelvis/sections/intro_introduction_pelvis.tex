\section{Introduction}
\label{sec:intro_introduction_pelvis}

The pelvis roll angle stands as a crucial kinematic parameter, widely employed in gait analysis, performance assessment, and clinical assessment of horses  \cite{SerraBraganca2018,Goff2018,Hardeman2019}. This parameter provides insights into gait coordination, helping identify asymmetries and irregularities that could impact movement efficiency. By optimizing training techniques based on pelvis roll angle analysis, performance can be enhanced and energy wastage reduced. Additionally, this measurement aids in injury prevention, rehabilitation planning, and early detection of musculoskeletal issues.

\gls{omc} has been the gold standard and the method of choice for calculating the pelvis roll angle. By attaching a reflective marker to each tuber coxae and measuring the displacement differences of the markers, the angle can be calculated. In general, using \gls{omc} as the measuring device confines assessments to laboratory environments and is not applicable to sport horses during field exercise. In contrast, \gls{imu}s provide a solution that is portable for the angle measurement. However, applying the same method of angle measurement to \gls{imu}s (attaching an \gls{imu} to each tuber coxae) proves to be more time-intensive, which results from a meticulous and expert approach when locating the precise anatomical site.

One potential approach involves employing a single \gls{imu} solely on the sacrum, the triangular bone at the base of the spine. The key advantage of this method is its minimalistic requirement for just one \gls{imu}, which can significantly reduce the complexity of data collection and analysis. Additionally, the process of locating and attaching the \gls{imu} to the sacrum is relatively straightforward when compared to the more intricate process of placement on the tuber coxae, a prominent hip bone.  

A few studies tried to calculate the pelvis roll angle using \gls{imu}s or for \gls{omc}, where they attached an \gls{imu} or a reflective marker attached to each tuber coxae \cite{Day2020,Hardeman2020,Greve2018,SerraBraganca2018,Roepstorff2009}. Although one study attempted to calculate this parameter using one \gls{imu} on the sacrum \cite{Pfau2013}, they did not compare the \gls{imu} results with the \gls{omc} system output, the gold standard of gait analysis \cite{Schmutz2020}. They assumed that the sacrum's three-dimensional location is precisely between the left and right tuber coxae \cite{Pilliner2009TheLocomotion}. 

The accuracy and reliability of using single \gls{imu} can also be measured using the Bland-Altman analysis. The Bland-Altman analysis, a fundamental methodology in quantitative research and scientific investigations, plays a pivotal role in assessing measurement agreement between two distinct techniques \cite{blandaltman}. This approach offers a valuable framework for comparing two measurement methods, often termed the "gold standard" and the "test" method. The Bland-Altman analysis facilitates the identification of any systematic bias or variability between these methods. 

In this chapter, we calculated the pelvis roll angle during walk and trot using one \gls{imu} on the sacrum and compared it to \gls{omc}. In addition, we compared the results to the outcome of the displacement difference method, where an \gls{imu} or \gls{omc} is placed on each tuber coxae. 
Furthermore, the Bland-Altman analysis method served as a valuable tool for assessing the accuracy and reliability of the pelvis roll calculation method. This assessment involved a comparative analysis of the pelvis roll calculation method using \gls{omc} and two \gls{imu}s each once as reference standards. According to the highly accurate results of the \gls{imu} validation study \cite{456}, we hypothesized a high agreement between \gls{imu} and \gls{omc}.
