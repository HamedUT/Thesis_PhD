\begin{quote}
    It is yet unknown if one \gls{imu} on the sacrum is adequate for calculating pelvis roll angle compared to the gold standard, \gls{omc}. Five horses were equipped with two measurement devices, \gls{imu}s and reflective markers on the sacrum and both tuber coxae. Then, they were measured during a straight walk and trot. Pelvis roll angle was calculated per stride using two methods per measurement device, the rotation angle of the markers or the \gls{imu} on the sacrum (roll method), and the rotation angle from the vertical displacement of \gls{imu}s and markers on tuber coxae (inverse sine method). Using Bland-Altman analysis, limits of agreement between methods were reported. Results demonstrated high agreements between the measurement devices within the methods, which validated the usability of \gls{imu} for calculating pelvis roll angle. Moreover, high biases between the methods with the same measurement devices emerged, which were likely due to the difference in the center of rotation, where the roll method was above and the inverse sine method was below the tuber sacrale. In conclusion, one \gls{imu} on the sacrum is sufficient for calculating pelvis roll angle; however, none of the methods indicated the actual pelvis roll angle around the sacrum.
\end{quote}

\blfootnote{This chapter is an extended version of the poster originally presented as \textit{'Comparing two methods of estimating pelvis roll using IMU sensors and optical motion capture'} in the International Conference on Equine Exercise Physiology in 2022.}