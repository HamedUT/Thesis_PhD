\section{Introduction}
\label{sec:intro_introduction_speed}


Speed is a key variable for analyzing horse locomotion and assessing the fitness level \cite{meira_2014_speed,witte_2004_accuracy}. Effects of speed on kinematic and kinetic parameters as well as health indicators have been studied in a variety of equine-related areas \cite{robert_2002_effects,weishaupt,allen_2015_evaluation}, including injury prevention during exercise \cite{williams_2019_the}, lameness detection \cite{moorman_2017_the,9216873}, fitness level evaluation \cite{witte_2006_effect,munsters_2014_exercise}, and genomics analyses \cite{meira_2014_speed,farries_2019_analysis}. Specifically, speed alteration is closely related to different health indicators and spatiotemporal, kinematics, and kinetics parameters, including stride related parameters \cite{robert_2002_effects,heglund_1988_speed,weishaupt,allen_2015_evaluation}, muscle activity \cite{robert_2002_effects}, limb force \cite{witte_2006_effect}, energy consumption \cite{heglund_1988_speed}, and heart rate \cite{kingston_2006_use,allen_2015_evaluation,bazzano_2016_application,fonseca_2010_the,vermeulen_2006_measurements,williams_2019_the,younes_2015_speed}, which may occur in various disciplines and competitions, such as thoroughbred horseracing \cite{kingston_2006_use,vermeulen_2006_measurements,spence_2012_speed,han_2020_selection}, eventing \cite{munsters_2014_exercise}, endurance \cite{marlin_2018_equine,parkes_2019_the,bennet_2018_fdration,nagy_2010_elimination,younes_2015_speed}, and showjumping \cite{bazzano_2016_application,Schmutz2020}. Therefore, equine speed assessment is important as it assists clinicians and equestrians to have a comprehensive view of a horse's health and fitness level.

Traditionally, horse speed has typically been assessed through observational methods and categorized using relative terms, such as 'slow' and 'fast' \cite{enschede_kwpn}. Describing locomotion parameters relatively might be influenced by expectation bias \cite{knigvonborstel_2011_towards} and is contingent upon the assessor's level of expertise \cite{articlevisual,GMEL2020102932}. To address this issue, speed should be quantitatively measured to allow for meaningful comparisons.

In the existing body of literature, various methods for speed measurement using specialized devices have been discussed. Traditionally, high-speed cameras have been a prevalent choice for this purpose \cite{fredricson_1980_the,ratzlaff_1985_the}. However, the use of high-speed cameras entails fixing the camera to a stable surface, necessitates careful scaling, and is limited in its field of view. Alternatively, three-dimensional \gls{omc} systems have gained recognition for speed assessment \cite{ericson_2020_the}. This method is often considered the 'gold standard' for quantifying gait kinematics \cite{pfau_2005_a}. It is noteworthy, however, that the use of \gls{omc} systems is associated with a higher cost, and their application is primarily confined to controlled laboratory environments \cite{5409556}. 

In contrast to the constraints associated with the methods discussed above, \gls{imu} offer the advantage of quantifying movement patterns both indoors and outdoors. \gls{imu} technology has found extensive applications in the analysis of human and animal motion, demonstrating the capability to accurately measure both acceleration and angular velocity. In principle, integrating the acceleration signal yields velocity. However, in practical applications, the cumulative effect of errors during the integration process may manifest as a progressively increasing deviation from the true value over time \cite{Fasel2017,Feigl2020,Brzostowski2018,Diez2018}.

Another portable device for speed measurement is \gls{gps}. This device has been validated for accurate speed measurement \cite{witte_2004_accuracy,varley_2012_validity,borresen_2009_the,roe_2017_validity,beato_2018_validity} in human \cite{scott_2016_the} and equine studies \cite{kingston_2006_use,bazzano_2016_application,farries_2019_analysis,fonseca_2010_the,parkes_2019_the,vermeulen_2006_measurements,williams_2019_the,han_2020_selection,best_2019_the}. The advantage of using \gls{gps} is portability and low price. However, it cannot be used indoors due to the lack of received signals from satellites. The accuracy might also be affected outdoors if large obstacles hinder the connection. 

To estimate horse speed effectively, a viable approach involves utilizing \gls{gps} speed as the ground-truth reference and \gls{imu} for its portability. However, it's important to note that \gls{imu} data is inherently high-dimensional, which introduces complexity when attempting to establish a connection with one-dimensional scalar speed data. One potential solution to this challenge lies in the application of machine learning techniques, which have demonstrated the ability to process non-linear and high-dimensional data, ultimately leading to the development of an optimized model for this purpose \cite{phinyomark2018analysis}. Therefore, this study aims to estimate horse speed using body-attached \gls{imu}s and machine learning methods. 

Most commercial wearable sensors for humans are designed to be worn on the wrist. This design choice is primarily made for convenience and accessibility. However, it may compromise the accuracy of collected data. For instance, using a chest strap for heart rate monitoring and a foot-mounted sensor for measuring running balance and cadence has been demonstrated to be more accurate than using wrist-worn sensors. Hence, in this study, the effects of the number and location of \gls{imu}s on the accuracy of the estimation were compared. Furthermore, to achieve an accurate and complete model, different machine learning approaches, gaits, and breeds (with distinct movement patterns) on the estimation model were studied.