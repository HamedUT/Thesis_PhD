\section{Conclusion}
\label{sec:conclusion}

In this chapter, three novel outcomes were achieved. First, the investigation for the optimal \gls{imu} placement concluded that placing only one \gls{imu} on any limb, withers, or sacrum results in a high accuracy of the speed estimation model. Second, the model can estimate the speed during distinct motion patterns from two different breeds and five gaits. Finally, the performances of five prevalent machine learning techniques on estimating the horse speed were compared. 

In total, 55 machine learning estimation models were trained, validated, and evaluated. Models based on the random forest method performed better than \gls{svm}, \gls{gpr}, decision tree, and \gls{gbt} methods. Moreover, the model can accurately estimate the speed during all gaits using the output signals of only one \gls{imu} (limb, sacrum, or withers). Therefore, this robust model offers flexibility to researchers, equestrians, and breeders in terms of \gls{imu} location on the horse's body, and thus, they can benefit from accurate speed estimation in training and competition. In future studies, model capability might be explored by adding data from different breeds or age-specific horses measurements. Performance of the model can also be investigated by training it on novel deep learning methods.