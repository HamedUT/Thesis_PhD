\begin{quote}
Speed is an essential parameter in biomechanical analysis and general locomotion research. It is possible to estimate the speed using \gls{gps} or \gls{imu}. However, \gls{gps} requires a consistent signal connection to satellites, and errors accumulate during \gls{imu} signals integration. In an attempt to overcome these issues, we have investigated the possibility of estimating the horse speed by developing machine learning models using the signals from seven body-mounted \gls{imu}. Since motion patterns extracted from \gls{imu} signals are different between breeds and gaits, we trained the models based on data from forty Icelandic and Franches-Montagnes horses during walk, trot, tölt, pace, and canter. In addition, we studied the estimation accuracy between \gls{imu} locations on the body (sacrum, withers, head, and limbs). The models were evaluated per gait and were compared between machine learning algorithms and \gls{imu} location. The model yielded the highest estimation accuracy of speed within equine and most of the human speed estimation literature. In conclusion, we have developed highly accurate machine learning models for estimating horse speed that are independent of the location of the \gls{imu} on the horse's body and the type of gait.
\end{quote}

\blfootnote{This chapter is a revised and extended version of the work originally published as \textit{'Using different combinations of body-mounted IMU sensors to estimate speed of horses—a machine learning approach'} in Sensors (MDPI) Journal on 2021.}