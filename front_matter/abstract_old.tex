\chapter*{Abstract}

This PhD thesis explores the use of inertial sensors as wearables on horses to improve their performance and prevent injuries. The aim of this research is to develop a system that can accurately monitor the horse's movements and provide real-time feedback to the rider or trainer, allowing them to make adjustments to improve performance and prevent injuries.

The study begins by reviewing the current literature on horse performance and injury prevention, as well as the different types of wearable sensors that have been used on horses. Based on this review, the research team selected inertial sensors as the most suitable wearable technology for the project due to their ability to capture a wide range of motion data in real-time.

The next phase of the research involved designing and testing a prototype sensor system that could be worn by horses during training and competition. This system consisted of multiple inertial sensors placed at strategic points on the horse's body, including the head, neck, shoulders, back, and legs. The sensors were able to capture data on the horse's movement patterns, speed, stride length, and other key performance metrics.

The prototype system was tested in a variety of training and competition scenarios, with data collected and analyzed to evaluate its effectiveness. The results showed that the system was able to accurately capture and analyze a wide range of motion data in real-time, providing valuable feedback to trainers and riders on how to improve performance and prevent injuries.

Overall, this PhD thesis demonstrates the potential of using inertial sensors as wearables on horses to improve their performance and prevent injuries. By providing real-time feedback on key performance metrics, this technology has the potential to revolutionize the way horses are trained and managed, leading to better performance, increased safety, and a more sustainable equestrian industry.
\mainrq







