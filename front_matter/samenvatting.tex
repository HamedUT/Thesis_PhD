\chapter*{Samenvatting}

Het vakgebied van menselijke sporten heeft opmerkelijke technologische vooruitgangen gezien, waarbij gebruik wordt gemaakt van machine learning en verschillende sensoren voor prestatieanalyse. Echter, het domein van sportpaarden blijft achter in technologische ontwikkeling. In tegenstelling tot mensen, missen paarden de mogelijkheid om gemakkelijk begrijpelijke feedback te geven, zoals verbale uitingen van vermoeidheid of het aangeven van de moeilijkheidsgraad van een trainingssessie.

Traditioneel hebben dierenartsen en onderzoekers methoden ontwikkeld om het welzijn van paarden te interpreteren, waaronder verbale aanmoediging, gezichtsuitdrukkingen en bloedmonsteranalyse. De eerste twee methoden zijn subjectief en afhankelijk van experts en laboratoriumomgevingen voor interpretatie. De laatste methode, hoewel informatief, is invasief en veroorzaakt stress en ongemak bij paarden tijdens het verzamelen van monsters. Het is ook omslachtig omdat het meerdere onderbrekingen van trainingssessies vereist voor monsterafname. Bovendien kunnen onnauwkeurige of onbetrouwbare fitnessparametervariabelen de basis van een effectief trainingsplan compromitteren, wat kan leiden tot negatieve uitkomsten zoals overtraining en blessures. Daarom is het cruciaal om een methode te implementeren die vergelijkbaar is met die in menselijke sporten, die feedback kan geven, en om een draagbaar meetapparaat te kiezen dat fitnessparameters nauwkeurig en betrouwbaar kan beoordelen. Dit apparaat moet zijn ontworpen voor gebruik in het veld, zodat beoordelingen buiten een laboratoriumomgeving mogelijk zijn.

Dit proefschrift heeft tot doel de training van sportpaarden te revolutioneren door het gebruik van inerti\"{e}le sensoren als draagbare technologie en de integratie van state-of-the-art machine learning om de prestaties van paarden te verbeteren en blessures te voorkomen. De studie ontvouwt zich in negen hoofdstukken binnen twee onderling verbonden delen, waarbij elk hoofdstuk een cruciaal onderdeel bijdraagt aan het overkoepelende doel van het verbeteren van de fitheid en het welzijn van paarden.

Elk hoofdstuk begint met een overzicht van de bestaande literatuur, waarin hiaten en uitdagingen in de fitheid en prestaties van paarden worden geïdentificeerd. Op basis van de reviews werden inerti\"{e}le sensoren geselecteerd als de meest geschikte technologie vanwege hun vermogen om een breed scala aan realtime bewegingsgegevens vast te leggen. De hoofdstukken richten zich vervolgens op het gebruik van het meetsysteem door het op het lichaam van het paard te plaatsen, inclusief het hoofd, de nek, schouders, rug en benen. De meetfase omvat verschillende trainings- en wedstrijdscenario's, waarbij gegevens worden verzameld en geanalyseerd om de effectiviteit van het systeem te evalueren. De resultaten tonen aan dat het systeem in staat was om nauwkeurig een breed spectrum aan bewegingsgegevens vast te leggen en te analyseren, waardoor waardevolle inzichten werden verkregen voor trainers en ruiters om de fitheid te verbeteren en blessures te voorkomen.

Dit proefschrift vertegenwoordigt een significante stap voorwaarts in paardensportonderzoek door gebruik te maken van draagbare sensortechnologie en machine learning om ons begrip van de fitheid, prestaties en welzijn van paarden te verbeteren. De kennis die uit deze hoofdstukken wordt verkregen, informeert niet alleen de wetenschappelijke gemeenschap, maar ook de bredere paardensportwereld, en biedt praktische hulpmiddelen voor het verbeteren van het welzijn van sportpaarden. Door feedback te geven via de evaluatie van fitheid tijdens trainingssessies, heeft deze technologie de potentie om de prestaties te verbeteren en bij te dragen aan de duurzaamheid van de paardensportindustrie.