\chapter*{Abstract}

The field of human sports has seen remarkable technological advancements, incorporating machine learning and various sensors for performance analysis. However, the domain of sport horses lags in technological development. Unlike humans, horses lack the ability to provide easily understandable feedback, such as verbal expressions of fatigue or conveying the difficulty of a training session. 

Conventionally, veterinarians and researchers have devised methods to interpret equine well-being, including verbal encouragement, facial expressions, and blood sample analysis. The former two methods are subjective, relying on experienced individuals and laboratory environments for interpretation. The latter, while informative, is invasive, inducing stress and discomfort in horses during sample collection. It is also cumbersome, as it necessitates multiple interruptions to training sessions for sample collection. Furthermore, inaccurate or unreliable fitness parameter values can compromise the foundation of an effective training plan, potentially resulting in adverse outcomes such as overtraining and injury. Therefore, it is crucial to implement a method akin to those used in human sports, capable of providing feedback, and to choose a portable measuring device that can accurately and reliably assess fitness metrics. This device should be designed for field use, enabling assessments outside of a laboratory setting.

This PhD thesis aims to revolutionize the training of sport horses by exploring the use of inertial sensors as wearable technology and the incorporation of state-of-the-art machine learning to enhance equine performance while preventing injuries. The study unfolds in nine chapters within two interconnected parts, with each contributing a crucial piece to the overarching goal of enhancing equine fitness and well-being.

Each chapter begins with a review of the existing literature, identifying gaps and challenges in equine fitness and performance. Based on the reviews, inertial sensors were selected as the most suitable technology for their ability to capture a wide range of real-time motion data. The chapters then focused on using the measurement system by placing it on the horse's body, including the head, neck, shoulders, back, and legs. The measurement phase involved various training and competition scenarios, with data collected and analyzed to evaluate the system's effectiveness. The results revealed the system's capacity to accurately capture and analyze a broad spectrum of motion data, providing valuable insights to trainers and riders for fitness improvement and injury prevention.

This thesis represents a significant stride in equine research, leveraging wearable sensor technology and machine learning to enhance our understanding of equine fitness, performance, and well-being. The knowledge gained from these chapters informs not only the scientific community but also the broader equestrian world, offering practical tools for improving the welfare of sport horses. By providing feedback through the evaluation of fitness during training sessions, this technology has the potential to enhance performance and contribute to the sustainability of the equestrian industry.






