\section{Objectives, Motivations, and Methodology}
\label{sec:Objectives}

Based on the introduction and the background of studies on equine fitness, several research gaps were identified. The primary objectives of this thesis are therefore formulated to systematically address and contribute to the resolution of these identified gaps. Each objective is supported by its corresponding motivational factor, followed by a proposed method for fulfillment. The following objectives, along with their respective motivational factors and proposed methodologies, outline the specific focus areas:
    \vspace{0.25cm}


    \noindent\textbf{Objective #1:} 
    Collection, preprocessing, and analysis of data to measure and identify relevant features and patterns associated with horse fitness.
    
    \textbf{Motivational factor: }It addresses a critical need for objective, accurate, and non-invasive methods to evaluate the fitness and assess the fatigue of sport horses, which can improve animal welfare, optimize training programs, and prevent injuries.
    
    \textbf{Proposed methodology: }First, we conduct a systematic literature review to identify relevant studies on the use of wearable sensors and machine learning in equine sports science to synthesize the current state of knowledge and future directions. Then, we design and implement a data collection protocol for gathering data from wearable sensors mounted on sport horses during motion and gait, while ensuring the quality and representativeness of the data. Lastly, we preprocess and analyze the collected data using appropriate techniques, such as signal processing, feature extraction, and dimensionality reduction, to identify relevant features and patterns associated with fitness.
    \vspace{0.25cm}
    
    \noindent\textbf{Objective #2:} Development and evaluation of machine learning models capable of accurately assessing horse fitness and predicting fatigue using the identified features and patterns.
    
    \textbf{Motivational factor: }It contributes to the growing body of research on the application of wearable sensors and machine learning in equine sports science, which could lead to new insights and technological advancements in the field.

    \textbf{Proposed methodology: }We develop, train, and evaluate machine learning models using the preprocessed data, by applying a range of algorithms (e.g., decision trees, support vector machines, neural networks) and model selection techniques (e.g., cross-validation, grid search) to identify the most suitable models. Various machine learning algorithms will be applied to the preprocessed data to develop predictive models for assessing the fitness and fatigue of sport horses. Techniques such as supervised learning, unsupervised learning, and deep learning will be explored and compared to determine the most suitable approach.
        \vspace{0.25cm}

    \noindent\textbf{Objective #3:} Validating the accuracy and reliability of the developed machine learning models using experimental data collected from horses during training and competition.
    
    \textbf{Motivational factor: }Similar to the motivational factor behind Objective 2, it contributes to the research domain of wearable sensors and machine learning, exploring various possibilities and forms of artificial intelligence in the context of fitness measurements.

    \textbf{Proposed methodology: }The accuracy and reliability of the developed machine learning models will be evaluated using experimental data collected from horses during training and competition. Metrics such as sensitivity, specificity, \gls{rmse}, and \gls{mape} will be used to assess the performance of the models.
        \vspace{0.25cm}

    \noindent\textbf{Objective #4:} Optimization and validation of the developed models for robustness and generalizability.
    
    \textbf{Motivational factor: }It offers practical solutions that can be readily implemented by trainers, veterinarians, and other professionals involved in the care and management of sport horses, potentially leading to better performance outcomes and healthier animals. 

    \textbf{Proposed methodology: }The models are tested and optimized in terms of resource allocation, sampling rate of the sensors, and sensors' mounting location on the body. Based on the results, practical guidelines for implementing these models in real-world settings are outlined.
        \vspace{0.25cm}

    \noindent\textbf{Objective #5:} Exploring the practical applications of the developed machine learning models and their integration into the daily routines of trainers, veterinarians, and riders.
    
    \textbf{Motivational factor: }The justification for this study lies in the potential benefits it can bring to various stakeholders, including:
    \vspace{-0.25cm}
	\begin{enumerate}
	    \item Horse owners and trainers, who can use the developed models to monitor their horses' fitness levels more accurately, make informed decisions about training and competition schedules and reduce the risk of injury.
        \item Veterinarians, who can use the proposed methods as an additional tool for diagnosing and treating fatigue-related health issues in sport horses, as well as monitoring their recovery and rehabilitation.
        \item Equestrian sports governing bodies, who can use the findings of this study to develop  evidence-based guidelines and regulations aimed at promoting the welfare and fitness of sport horses.
        \item Researchers in the fields of equine sports science and veterinary medicine, who can build upon the methods and findings of this study to advance the understanding of horse biomechanics, physiology, and health.
\end{enumerate}
\vspace{-0.2cm}
\textbf{Proposed methodology: }The practical applications of the developed machine learning models will be explored, and their integration into the daily routines of trainers, veterinarians, and riders will be discussed. This will include recommendations for the use of the models in training, fitness management, and injury prevention.

    \vspace{0.25cm}

The scope of this study is limited to sport horses participating in endurance racing, dressage, show jumping, and eventing. The chosen disciplines were selected to address prevailing research gaps within the equestrian literature. Each discipline presents a unique set of physical and mental challenges for sport horses, contributing to a comprehensive understanding of their performance and well-being. The practical relevance of these disciplines is underscored by their widespread popularity and economic significance within the equestrian community. Moreover, the diverse nature of these disciplines ensures a holistic investigation into the multifaceted aspects of equine athleticism. The exclusion of other horse disciplines is a deliberate choice aimed at maintaining the research's focus and ensuring a thorough examination of the selected disciplines. 

The study focuses on the use of \gls{imu}, \gls{hr} monitors, blood \gls{lac} meters, and \gls{gps} trackers for data collection, and machine learning algorithms for data analysis and model development. The research will focus on developing machine learning models and will not cover other aspects of equine sports science, such as nutrition, psychology, or genetics.

Throughout the study, ethical considerations will be taken into account, ensuring the welfare of the horses involved in the data collection process and adhering to relevant guidelines and regulations applicable to the research context.