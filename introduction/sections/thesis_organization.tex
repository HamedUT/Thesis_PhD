\section{Thesis Organization}
\label{sec:intro_thesis_organization}

The structure of this thesis consists of nine chapters, outlined as follows:

%\chaptername~\ref{chapter:state_of_the_art} introduces the process of \gls{aar}, its intermediate steps, and a survey of the state of the art.
\vspace{0.25cm}

\textbf{Chapter 1: General Introduction} - Provides an overview of the study, including the background and context, research problem and questions, objectives and scope, significance and justification, methodology, and organization of the thesis.\\

\vspace{0.5cm}

\noindent \textbf{\large Part one: Data} 

\vspace{0.25cm}

\textbf{Chapter 2: Data Acquisition} - Describes the data collection protocol, including the selection of subjects, inertial sensors, exercise tasks, and data acquisition procedures, as well as any challenges encountered and solutions implemented during the process.\\

\vspace{0.5cm}


\textbf{Chapter 9: Conclusions and Future Work} - Summarizes the main findings of the study, discusses their implications for various stakeholders, and provides recommendations for future research in the field.\\

By addressing the research problem and objectives outlined in this introduction, this thesis aims to make a significant contribution to the fields of equine sports science, machine learning, and wearable technology utilization, ultimately improving the welfare of sport horses.


