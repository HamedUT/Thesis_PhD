\section{Motivation}
\label{sec:motivation}

This study is driven by several motivating factors:
\begin{enumerate}
    \item It addresses a critical need for objective, accurate, and non-invasive methods to evaluate the fitness and assess the fatigue of sport horses, which can improve animal welfare, optimize training programs, and prevent injuries.
    \item It contributes to the growing body of research on the application of wearable sensors and machine learning in equine sports science, which could lead to new insights and technological advancements in the field.
    \item It offers practical solutions that can be readily implemented by trainers, veterinarians, and other professionals involved in the care and management of sport horses, potentially leading to better performance outcomes and healthier animals. 
    \item The justification for this study lies in the potential benefits it can bring to various stakeholders, including:
	\begin{enumerate}
	    \item Horse owners and trainers, who can use the developed models to monitor their horses' fitness levels more accurately, make informed decisions about training and competition schedules and reduce the risk of injury.
        \item Veterinarians, who can use the proposed methods as an additional tool for diagnosing and treating fatigue-related health issues in sport horses, as well as monitoring their recovery and rehabilitation.
        \item Equestrian sports governing bodies, who can use the findings of this study to inform the development of evidence-based guidelines and regulations aimed at promoting the welfare and fitness of sport horses.
        \item Researchers in the fields of equine sports science and veterinary medicine, who can build upon the methods and findings of this study to advance the understanding of horse biomechanics, physiology, and health.
\end{enumerate}
\end{enumerate}