\section{Introduction}
\label{sec:intro_introduction}

Sport horses are exceptional athletes that have been specifically bred and trained for various equestrian disciplines. These magnificent quadrupeds possess a unique combination of strength and agility, making them ideal partners for competitive sports such as show jumping, dressage, eventing, and endurance riding \cite{BARTOLOME}. They have undergone rigorous training regimes for the development of their physical abilities and the improvement of their skills, ensuring they can perform at the highest levels of competition. Occasionally, they are pushed beyond their physical and physiological limits, resulting in a  state known as overtraining. As a consequence, overtraining can lead to fatigue, which not only impairs fitness but also jeopardizes overall health and well-being \cite{takahashi2004}. 

Within the domain of sports, fatigue is frequently described as the feeling of exhaustion coupled with a decline in athletic fitness. Looking at it from a scientific standpoint, fatigue is a multifaceted physiological response to exercise that hinders the ability to maintain the current level of activity \cite{MAMI201974}. When fatigue sets in, horses either cease exercising altogether or continue at lower intensities \cite{GERARD201419}.  

One of the main consequences of fatigue in sport horses is an increased risk of injury \cite{yoshikawa}. As a horse's energy levels decrease, their coordination and muscle control become compromised, making them more prone to mishaps. Fatigued horses may stumble or misjudge distances, leading to falls or collisions. These incidents can result in severe injuries, ranging from sprains and strains to fractures or even more catastrophic outcomes \cite{MAMI201974}. In severe cases, fatigue can cause horses to collapse and result in sudden death during competitions \cite{verheyenarticle}.

The fatigue effects can extend beyond physical injuries, as the psychological impact can lead to diminished confidence and increased anxiety in future performances \cite{BARTOLOME}. This can manifest in various behavioral issues such as resistance to training, irritability, and a lack of focus during competitions. Fatigue also compromises a horse's immune system, making them more vulnerable to infections and illnesses. As a result, fatigued horses may take longer to recover from common ailments and become more susceptible to long-term health issues \cite{MAMI201974}.

Recognizing the signs of fatigue is crucial to prevent injuries in sport horses. The monitoring of fitness levels during exercises and the implementation of preventive measures can be helpful in recognizing and addressing early signs of fatigue or underlying health issues. With a diligent approach to recognizing fatigue and proactive steps taken to mitigate its effects, optimal conditions can be maintained for sport horses to sustain peak fitness and overall well-being.

%This PhD thesis aims to investigate the potential of combining inertial sensors with machine learning algorithms for performance evaluation and fatigue prediction of sport horses. The research will contribute to the fields of equine sports science, machine learning, and wearable technology by providing insights into the complex biomechanics of horses and developing predictive models to enhance performance.






